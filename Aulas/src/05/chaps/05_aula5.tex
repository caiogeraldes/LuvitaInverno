% TeX root=../main.tex

\chapter{Leitura: \logo{KARATEPE}}

\clearpage

\begin{parnumbersr}

	\raggedright%
	\itshape%

	\lmasc{}\logo{EGO}-mi
	\spac{}\logo{(LITUUS)}á-za-ti-i-wa/i-da-sá
	\logo{(DEUS)}\logo{SOL}-mi-sá
	\logo{(CAPUT)}-ti-i-sá
	\logo{(DEUS)}\logo{TONITRUS}-hu-ta-sa
	\logo{SERVUS}-la/i-sá

	á-wa/i+ra/i-ku-sa-wa/i
	\lbreak{}
	\logo{REL}-i-na
	\logo{MAGNUS}+ra/i-nu-wa-ta
	á-TANA-wa/i-ní-i-sá\logo{(URBS)}
	\logo{REX}-ti-sá

	wa/i-mu-u
	\logo{(DEUS).TONITRUS}-hu-za-sa á-TANA-wa/i-\emph{||}-ia\logo{(URBS)}
	\logo{MATER}-na-tí-na tá-ti-ha i-zi-i-da

	\lmasc{}\logo{ARHA}-ha-wa/i
	\lmasc{}la+ra/i+a-nú-ha
	\lmasc{}á-TANA-wa/i-na\logo{(URBS)}

	\lmasc{}\logo{(“MANUS”)}la-tara/i-ha-ha-wá/í
	\lmasc{}á-TANA-wá/í-za\logo{(URBS)} \lmasc{}\logo{“TERRA+X”}{(-)}wá/í+ra/i-za
	\lmasc{}zi-na \lmasc{}\logo{(“OCCIDENS”)}i-pa-mi \lmasc{}\logo{VERSUS}-ia-na
	\lmasc{}zi-pa-wá/í \logo{(ORIENS)}ki-sà-ta-mi-i \lmasc{}VERSUS-na

	\lmasc{}á-mi-ia-za-há-wa/i \logo{(“DIES<”>)}ha-lí-za
	\lmasc{}á-TANA-wá/í-ia\logo{(URBS)} \lmasc{}\logo{OMNIS+}MI-ma
	\logo{(“BONUS”)}sa-na-wa/i-ia \lmasc{}\logo{(“CORNU+RA/I”)}su+ra/i-sa
	\lmasc{}\logo{(LINGERE)}ha-sa-sa-ha á-sá-ta


	\lmasc{}\logo{(“MANUS”)}su-wá/í-ha-ha-wá/í
	\lmasc{}pa-há+ra/i-wa/i-ní-zi\logo{(URBS)}
	\lmasc{}\logo{(<“>*255”)}ka-ru-na-zi

	\lmasc{}\logo{EQUUS.ANIMA}-zú-ha-wa/i-ta \logo{(EQUUS.ANIMA)}á-zú-wa/i
	\lmasc{}\logo{SUPER}+ra/i-ta \lmasc{}i-zi-i-ha

	\logo{EXERCITUS}-lu/a/i-za-pa-wa/i-ta \lmasc{}\logo{EXERCITUS}-lu/a/i-ní
	\lmasc{}\logo{SUPER}+ra/i-ta \lmasc{}i-zi-i-há

	\lmasc{}\logo{(<“>SCUTUM”)}hara/i-li-pa-wa/i-ta
	\lmasc{}\logo{(“SCUTUM”)}hara/i-li \lmasc{}\logo{SUPER}+ra/i-ta
	\lmasc{}i-zi-i-há $[$\logo{OMNIS-}MI-ma-za
					\lmasc{}\logo{{(DEUS)}TONITRUS}-hu-ta-tí \logo{DEUS}-na-ri+i-ha$]$


\end{parnumbersr}


\vspace{10pt}
\hrule
\vspace{10pt}


\setcounter{parcount}{0}
\begin{parnumbersr}

	\raggedright%
	\itshape%

	amu=mi Azatiwadas tiwadamis \logo{CAPUT}-tis Tarhunt{(a)}s hudarlis,

	Awarikus=wa \lbreak{} kwin uranuwata Adanawanis hantawatis,

	*a=wa=mu Tarhunz Adanawaya anatin tadin=ha izida.

	arha=ha=wa laranuha Adanawan.

	lataraha=ha=wa Adanawan=za walirin=za zin ipami tawiyan zin=pa=wa kistami
	tawiyan.

	amiyanza=ha=wa halinza Adanawaya tanima sanawiya \logo{(“CORNU+RA/I”)}-suras
	hasas=ha asta.

	suwaha=ha=wa Paharawaninzi karunanzi,

	azun=ha=wa=ta azuwi sara iziha,

	kulanin=za=pa=wa kulani sara iziha,

	haralin=pa=wa=ta harali sara iziha, taniman=za Tarhuntadi masanari=ha.


\end{parnumbersr}


\clearpage%
\noindent\textbf{Tradução}

[I] Eu sou Azatiwada, homem abençoado{(?)} pelo sol, servo de Tarhunta, [II] que
Awariku, rei de Adanawa, elevou, [III] e Tarhunta me fez da (cidade de)
Adanawa mãe e pai. [IV] Eu fiz (a cidade de) Adanawa prosperar, [V] eu estendi
a planície de Adanawa de um lado em direção ao ocidente, do outro em direção
ao oriente [VI] e, nos meus dias, havia em Adanawa todos os bens, abundância e
saciedade (\emph{ou} luxo). [VII] Eu enchi os silos de Pahara [VIII] e fiz
cavalo e mais cavalo, [IX] e fiz exército e mais exército, [X] e fiz escudo e
mais escudo, tudo por {(graça de?)} Tarhunta e pelos deuses (\emph{ou} pela
graça dos deuses).

\bigskip
\noindent\textbf{Notas}

\smallskip
\noindent\textbf{§I}\tabto{2em}
\textbf{\emph{tiwadamis}} `abençoado/a pelo deus Sol': o nome do deus Sol em
luvita é \emph{Tiwad{(a)}-}\footnote{Ver formas quase completas em KÜRTÜL, §6 e
	KARKAMIŠ A15\emph{b}, §1.} e esta forma utiliza o sufixo de formação de
adjetivos \emph{-ami-}.
O sentido específico deo adjetivo como \emph{abençoado/a} é gerado a partir do
fenício \emph{h-brk}.
\textbf{CAPUT-\emph{tis}} `pessoa, homem': a forma subjacente é incerta, nunca
sendo escrita em sua completude fonologicamente.
O termo \emph{ziti-} `homem' parece apenas ocorrer com L.313 𔕠 VIR, fazendo-nos
crer que L.10 𔐉 CAPUT é reservada para outro elemento semântico.
No entanto, em diversas passagens de KARATEPE, CAPUT\emph{-ti-} corresponde ao
fenício \emph{ʾdm} `homem'.
\textbf{\emph{hudarlis}} `servo': fonologia reconstruída a partir do
luvita cuneiforme \emph{hudarli-}.

\smallskip
\noindent\textbf{§II}\tabto{2em}
\textbf{\emph{Awarikus=wa kwin}}: oração relativa com o sujeito antecedendo o
pronome que recupera \emph{amu} `eu' de §1. Awariku foi por vezes
identificado com o rei Urikki de Que, tributário de Tiglate-pileser III\@,
mas a evidência é pouca e há a possibilidade de ser o avô deste.

\smallskip
\noindent\textbf{§III}\tabto{2em}
\textbf{\emph{Tarhunz}} `Tarhunta': a divindade Tarhunta no texto fenício é
traduzida como \emph{bʿl} `Baal \slash{} senhor'.
A variação das formas \emph{\logo{{(DEUS)}TONITRUS}-hu-ta-sa} e
\emph{\logo{{(DEUS)}TONITRUS}-hu-za} talvez indique que o nome da divindade fosse
um tema consonantal em \emph{-t}, \emph{Tarhunt-}, com o nominativo
\emph{Tarhunz} \ipa{/tar.hunts/}, o mesmo valendo para a divindade solar Tiwad,
cujo nominativo seria \emph{Tiwaz} \ipa{/ti.wats/}.
\textbf{\emph{\emph{MATER}-na-tí-na}} `mãe': a leitura é garantida pelo fenício
\emph{ʾm} `mãe', pois a partir da grafia luvita, tanto \emph{anatin} `mãe'
quanto \emph{wanatin} `mulher' poderiam ser interpretados, uma vez que L.79 𔑘
FEMINA\slash{}MATER é utilizado para ambos os temas e ambos são temas em
\emph{-n-} sufixadas pelo morfema \emph{-ati-}.

\smallskip
\noindent\textbf{§IV}\tabto{2em}
\textbf{\emph{ARHA} {=?} \emph{arha-/aha-}} `completamente': é incerto se o
prevébio e advérbio representado por L.216 𔓸 ARHA era fonologicamente realizado
com a sequência \ipa{/rh/} ou com a sequência \ipa{/hh/} produzida por
assimilação, \emph{vide} hit.\ \emph{\hittitetrans{arha}} mas luv.cun.
\emph{\hittitetrans{ahha}}. Para uma discussão das formas,
ver~\citet{Yakubovich2012}.
\textbf{\emph{la+ra/i+a-nú-ha} {=?} \emph{laranuha}} `fazer prosperar?': talvez
seja uma forma causativa do verbo \emph{lada-\slash{}lara-} atestado em
AKSARAY, §2 e SULTANHAN, §6. O sentido é produzido a partir da comparação com o
hit.\ \emph{lazziya-} `prosperar', embora não esteja clara a fonologia.
A passagem em fenício contém \emph{ḥw} `fazer viver'.
Ver mais em~\citet[104--5]{HawkinsMorpurgo1978}.

\smallskip
\noindent\textbf{§V}\tabto{2em}
\textbf{\emph{\emph{“TERRA+X”}{(-)}wá/í+ra/i-za} {=?} walirin=za} `planície':
para discussão sobre a forma subjacente e troca da forma esperada
\emph{walili{(da)}-} por \emph{waliri{(da)}-},
ver~\citet[106]{HawkinsMorpurgo1978}, que também sugerem a possibilidade de
uma haplologia, i.e. *\emph{walirin=za} >
\emph{warin=za}, ou de haplografia, i.e.\ \luwiantrans{wá-ra-ra-za}
\emph{wá/í+ra/i-ra/i-za} > \luwiantrans{wá-ra-za} \emph{wá/í+ra/i-ra/i-za}.
Possível correlato de hit.\ \emph{ulili-} `campo'.
O sentido de `planície' é dado pelo fenício
\emph{ʿmq} `vale, planície'.
\textbf{\emph{zin\ldots{} zin=pa}} `de um lado\ldots{} do outro': o
ablativo-instrumental \emph{zin} tem o sentido de `aqui', a construção
contrastiva \emph{zin\ldots{} zin{(=pa)}} é comum para denotar `por um
lado\ldots{} por outro', no sentido local mas também lógico.


\smallskip
\noindent\textbf{§VI}\tabto{2em}
\textbf{\emph{tanima}} `todas': neutro plural de \emph{tanima-}, a forma neutra
plural em \emph{-aya} aparece em Ho.\ §XV\@.
\textbf{\emph{sanawiya}} `(coisas) boas = bens': neutro com sentido abstrato. A
interpretação da forma talvez seja \emph{sana-awi-} `bem-vindo', \emph{vide}
\citet{YakubovichWelcome}. Em fenício temos \emph{nʿm} `bens'.
\textbf{\emph{\emph{(“CORNU+RA/I”)}su+ra/i-sa} {=?} \emph{suras}} `abundância':
a forma subjacente não é cla\-ra, mas possivelmente esteja associada ao verbo
\emph{suwa-} `encher, preencher' (hit.\ \emph{suwai-}). A forma fenícia oferece o
sentido, \emph{šbʿ} `abundância'.
\textbf{\emph{\emph{(LINGERE)}ha-sa-sa} {=?} \emph{hasas}} `saciedade': a forma
subjacente é incerta, mas possivel\-mente seja um homônimo de \emph{hasa-} `força'
(KARKAMIŠ A11b+c, §30), que, no entanto, é acompanhada do logograma L.314 𔕡.
O logograma L.112 𔒈 LINGERE é sempre complementado por \emph{ha/há-sa/sá} e tem
o sentido de `saciedade' ou `luxo'.
O texto fenício apresenta \emph{mnʿm} `luxo'.

\smallskip
\noindent\textbf{§VIII-X}\tabto{4em}
\textbf{\emph{azun\slash{}kulanin\slash{}haralin\ldots{} azuwi\slash{}kulani\slash{}harali sara}}
`cavalo\slash{}exér\-ci\-to\slash{}escudo sobre cavalo\slash{}exército\slash{}escudo':
literalmente, as frases significam `eu fiz X sobre X', mas o sentido parece ser
de acúmulo `eu fiz X e mais X'.
Note-se que o texto fenício inverte a ordem de \emph{exército} e \emph{escudo},
Phoen. §IX \emph{mgn} `escudo' e §X \emph{mḥnt} `exército'.
O mesmo ocorre na versão hieroglífica Ho.

\smallskip
\noindent\textbf{§IX}\tabto{2em}
\textbf{\emph{\emph{EXERCITUS}-lu/a/i-za} {=?} \emph{kulanin=za}} `exército': se
aceitarmos que a for\-ma é idêntica ao luv.cun. \emph{kulana} (hit.
\emph{kuwalana-}), a melhor transliteração seria
\emph{\emph{EXERCITUS}+LU/A/I-za}, indicando que \emph{lu/a/i} age como
desambiguador fonológico e não se grafou a fonologia completa do termo.
Há também a possibilidade de se interpretar a forma subjacente como um tema em
nasal \emph{kulan-}, reforçado pela forma de ablativo
\emph{\emph{EXERCITUS}-lu/a/i-na-ti-i} {=?} \emph{kulanadi} (TELL AHMAR 6,
§24).

\smallskip
\noindent\textbf{§X}\tabto{2em}
\textbf{\emph{\emph{OMNIS}-MI-ma-za\ldots{} \emph{DEUS}-na-ri+i-ha}}: este trecho
está danificado em Hu., tendo sido reconstruído a partir da versão hieroglífica Ho.

\clearpage
\setcounter{parcount}{10}
\begin{parnumbersr}

	\raggedright%
	\itshape%


	\logo{REL}-pa-wá/í \lmasc{}\logo{(*255)}mara/i\textsuperscript{+ra/i}-ia-ní-zi
	\lmasc{}\logo{ARHA} \lmasc{}ma-ki-sa\textsuperscript{!}-há \lmasc{}\lmasc{}


	\lmasc{}\logo{(“MALUS2”)}ha-ní-ia-ta-<ia>-pa-wa/i-ta-a \lmasc{}\logo{REL}-ia
	\lmasc{}\logo{(TERRA)}ta-sà-REL+ra/i \lmasc{}a-ta \lmasc{}á-sá-ta


	\lmasc{}wá/í-ta \logo{(TERRA)}ta-sà-\logo{REL}+ra/i<-ri+i>
	\logo{ARHA} \lmasc{}\textsc{⌈}\logo{*501}\textsc{⌉} $[$…$]$-há

	\lmasc{}á-ma\lmasc{}\lmasc{}-za\textsubscript{4}-há-wá/í-ta
	\lmasc{}\logo{DOMINUS}-ní-za
	\lmasc{}\logo{DOMUS}-na-za \lmasc{}\logo{(BONUS)}sa-na-wá/í
	\lmasc{}u-sa-nú-há

	\lmasc{}á-mi-há-wa/i \lmasc{}\logo{DOMINUS}-ní-i
	\lmasc{}\logo{(NEPOS)}ha-su-a \lmasc{}\logo{OMNIS}-MI-ma
	\lmasc{}\lmasc{} \logo{(BONUS)}sa-na-wa/i-ia
	\lmasc{}\logo{CUM}-na i-zi-i-há

	\lmasc{}á-pa-sá-há-wá/í-ta \lmasc{}tá-ti-i
	\lmasc{}\logo{(“THRONUS”)}i-sà-tara/i-ti
	\lmasc{}\logo{(“SOLIUM”)}$[$i$]$-s$[$à-nu-wa/i-ha$]$

	$[$\ldots$]$

	$[$\lmasc{}\logo{OMNIS}-MI-sa-ha-wa/i-mu-ti-i \logo{REX}-ti-sa
					\lmasc{}tá-ti-na$]$ \lmasc{}$[$i-zi$]$-i-$[$da$]$
	\lmasc{}á-$[$mi$]$-ia-ti
	\lmasc{}\logo{IUSTITIA}-na-ti \lmasc{}á-mi-ia+ra/i-ha
	\lmasc{}\logo{(“COR”)}á-ta-na-sa-ma-ti \lmasc{}á-mi-ia+ra/i-há
	\lmasc{}\lmasc{} \lmasc{}\logo{(“BONUS”)}sa-na-wa/i-sa-tara/i-ti



\end{parnumbersr}

\vspace{10pt}
\hrule
\vspace{10pt}

\setcounter{parcount}{10}
\begin{parnumbersr}

	\raggedright%
	\itshape%

	kwipa=wa mariyaninzi arha makisaha,

	haniyataya=pa=wa=ta kwiya taskwiri anta asanta,

	a=wa=ta taskwirari arha parhaha.

	aman=za=ha=wa nanin=za parnan=za sanawi usanuha.

	ami=ha=wa nani \logo{NEPOS}-hasu{(w)}a tanima sanawaya \logo{CUM}-na iziha

	apasa=ha=wa=ta tati isatarati isanuwaha.

	$[$\ldots{}$]$

	tanimis=ha=wa=mu=ti hantawatis tadin izida amiyadi tarawanadi amiyari=ha
	atnasamadi amiyari=ha sanawastradi.

\end{parnumbersr}

\bigskip%
\noindent\textbf{Tradução}

[11] De fato fiz acumularem muito as colheitas dos campos-\emph{mariyana-}, [12]
enquanto os males que haviam na terra [13] eu os afastei completamente.
	[14] e a casa do meu senhor eu abençoei bem, [15] e fiz todos bens para a
descendência{(?)} do meu senhor, [16] e fi-lo sentar no trono paterno. [15]
\ldots{} [16] Todo rei me fez para si seu pai pela minha justiça e pela minha
sabedoria e pela minha bondade.

\bigskip
\noindent\textbf{Notas}

\smallskip
\noindent\textbf{§XI}\tabto{2em}
\textbf{\emph{mariyaninzi\ldots{} makisaha}} `acumulei
colheitas dos campos-\emph{mariyana}': a interpretação dessa passagem
é difícil, em parte pela presença de \emph{hapax legomena} tanto no texto luvita
quanto no texto fenício.
Sigo aqui a interpretação de \citet{VanDenHout2010}:
\textbf{\emph{mariyaninzi}}: ligada ao hitita
\textsuperscript{A.ŠÀ}\emph{mariyana-} `tipo de
campo? campo de um vegetal específico?' (KBo 10.37 12-17, 21-26), bem como às
formas luv.hier. \emph{mara/iwali-} `vegetação útil? centeio?' (SULTANHAN §6),
hit. \emph{marawalliya/i-} `campo de grãos', utilizando como evidência o uso de
L.255 𔔡 como determinativo de \emph{karunanzi} `silos (de grãos)' nesta
inscrição;
a forma escrita no texto, \emph{mariyaninzi}, deve ser interpretada como uma
forma contrata de *\emph{mariyaniyinzi}, contração da sequência \emph{-iyi-},
comum em luvita.
\textbf{\emph{makisaha}}: ligada ao hitita \emph{mekki-} `muito, numeroso' e à
passagem \emph{nu=kan ḫalkiuš EGIR-an maknunun} `eu fiz as colheitas (serem)
abundantes novamente' (Proclamação de Telipinu, KBo 3.1 iii 44, KUB 11.1 iii 8,
KBo 3.67 iii 1 + KUB 31.17:5).
Em resumo, a forma hipotética \emph{mariyaniyi-} significaria `relativo aos
campos do tipo \emph{mariyana-} > colheitas do campo-\emph{mariyana-}?' e o
verbo \emph{makisa-} seria uma forma iterativa de um verbo \emph{maki-} `fazer
crescer/abundante'.


\smallskip
\noindent\textbf{§XII}\tabto{2em}
\textbf{\emph{\emph{(“MALUS2”)}ha-ní-ia-ta-<ia>}} `males': o texto da versão
luvita Hu.\ parece ter ignorado um grafema, <\emph{ia}>, suplementado por conta
da versão Ho.\ e do pronome relativo \emph{kwiya} (nom.neut.pl.).

\smallskip
\noindent\textbf{§XIV}\tabto{2em}
\textbf{\emph{usanuwa}} `abençoar': literalmente, o verbo \emph{usanu{(wa)}-}
seria um causativo do verbo \emph{wasa-} `ser bom', logo `fazer ser
bom'. O sentido de abençoar neste contexto foi proposto pelo fato de que ao
longo do bilíngue, o fenício \emph{brk} `abençoar' é utilizado para traduzir
formas do verbo \emph{usanu{(wa)}-}.
No entanto, o texto fenício neste contexto contém o verbo \emph{yṭnʾ} `eu
ergui', o que suscitou as tentativa de interpretar \emph{usanu-} como um tema
cognato do hitita \emph{wete-} `construir', mas isso produziria um \emph{hapax
	legomena}.

\smallskip
\noindent\textbf{§XV}\tabto{2em}
\textbf{\emph{\emph{NEPOS}-hasu{(w)}a}} `descendência?': incerto, mas deve ser
um dativo singular comum.
