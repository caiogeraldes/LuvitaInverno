%! TeX program = lualatex
\documentclass[a4paper,12pt,article]{memoir}

% TeX root=../main.tex

\settrimmedsize{\stockheight}{\stockwidth}{*}
\settypeblocksize{220mm}{130mm}{*}
\setlrmargins{*}{*}{1.7}
\setulmargins{30mm}{*}{*}
\setmarginnotes{20pt}{100pt}{10pt}
\checkandfixthelayout%

% \setsidefeet{\marginparsep}{\marginparwidth}%
% {0.8\onelineskip}{0pt}%
% {\normalfont\footnotesize}{\textheight}%
\setsidecaps{\marginparsep}{\marginparwidth}
\setlength{\footmarkwidth}{0.5em}
\setlength{\footmarksep}{0em}
\setlength{\footparindent}{0em}
\footmarkstyle{\textsuperscript{#1}\hspace{0.5em}}

\makeoddfoot{plain}{}{}{\thepage}
\makeevenfoot{plain}{\thepage}{}{}
\makepagestyle{ruled}
\makeevenfoot{ruled}{\thepage}{}{} % page numbers at the outside
\makeoddfoot{ruled}{}{}{\thepage}
\makeheadrule{ruled}{\textwidth}{0.75pt}
\makeevenhead{ruled}{\scshape\leftmark}{}{}
\makeoddhead{ruled}{}{}{\scshape\rightmark}
\makepsmarks{ruled}{%
	\nouppercaseheads%
	\createmark{chapter}{left}{shownumber}{\scshape}{.\space}
	\createmark{part}{right}{shownumber}{}{.\space}
	\createmark{section}{right}{shownumber}{}{.\space}
	\createmark{subsection}{right}{shownumber}{}{.\space}
	\createplainmark{toc}{both}{\contentsname}
	\createplainmark{lof}{both}{\listfigurename}
	\createplainmark{lot}{both}{\listtablename}
	\createplainmark{bib}{both}{\bibname}
	\createplainmark{index}{both}{\indexname}
	\createplainmark{glossary}{both}{\glossaryname}
}

% decorando divisões
\setsecnumdepth{subsection}
\setcounter{tocdepth}{3}
\newcommand\chap[1]{%
	\chapter*[#1]{#1}%
	\addcontentsline{toc}{chapter}{#1}}


% paleta de cores
\usepackage{xcolor}
\definecolor{green}{rgb}{16,87,87} % rgb(16,87,87)
\definecolor{red}{rgb}{193, 11, 105} % rgb(193, 11, 105)
\definecolor{yellow}{rgb}{218,222,104} % rgb(218,222,104)
\definecolor{pink}{rgb}{243,179,145} % rgb(243,179,145)
\definecolor{blue}{rgb}{161,184,206} % rgb(161,184,206)

\usepackage[tracking=true]{microtype}

\usepackage{scalefnt}


\usepackage[normalem]{ulem}
\usepackage{amsmath}
\usepackage[defblank]{paralist}
\usepackage{graphicx}
\usepackage{subcaption}
\usepackage{linguex}
\usepackage{multicol}
\usepackage{multirow}
\usepackage{tabto}

\usepackage{hyperref}
\hypersetup{%
	colorlinks=true, % false: boxed links; true: colored links
	linkcolor=green,  % color of internal links
	citecolor=green,  % color of links to bibliography
	filecolor=pink,  % color of file links
	urlcolor=green,
}
% Configurações para o autoref
\renewcommand{\figureautorefname}{Figura}
\renewcommand{\tableautorefname}{Tabela}
\renewcommand{\sectionautorefname}{Seção}
\renewcommand{\chapterautorefname}{Capítulo}
\renewcommand{\subsectionautorefname}{Subseção}


\directlua{dofile('./utils/pie.lua')}
\newcommand{\hittitetrans}[1]{%
  \directlua{hittite_transcription("#1")}}
\newcommand{\luwiantrans}[1]{%
  \directlua{luwian_transcription("#1")}}
\newcommand{\pietrans}[1]{%
  \directlua{pie_transcription("#1")}}

\newcommand{\Prep}{{\footnotesize\textsc{Prep.}}}
\newcommand{\Det}{{\footnotesize\textsc{Det.}}}
\newcommand{\Clt}{{\footnotesize\textsc{Clt.}}}
\newcommand{\Nom}{{\footnotesize\textsc{Nom.}}}
\newcommand{\Acu}{{\footnotesize\textsc{Acu.}}}
\newcommand{\Dat}{{\footnotesize\textsc{Dat.}}}
\newcommand{\Gen}{{\footnotesize\textsc{Gen.}}}
\newcommand{\Abl}{{\footnotesize\textsc{Abl.}}}
\newcommand{\Sg}{{\footnotesize\textsc{Sg.}}}
\newcommand{\Pl}{{\footnotesize\textsc{Pl.}}}
\newcommand{\Com}{{\footnotesize\textsc{Com.}}}
\newcommand{\Neut}{{\footnotesize\textsc{Neut.}}}
\newcommand{\Rel}{{\footnotesize\textsc{Rel.}}}
\newcommand{\Conj}{{\footnotesize\textsc{Conj.}}}
\newcommand{\Pret}{{\footnotesize\textsc{Pret.}}}
\newcommand{\Pro}{{\footnotesize\textsc{Pro.}}}
\newcommand{\Refl}{{\footnotesize\textsc{Refl.}}}
\newcommand{\La}[1]{\textsc{l}.#1}
\newcommand{\logo}[1]{\textnormal{#1}}
\newcommand{\spac}{\textsuperscript{\textnormal{I}}}
\newcommand{\lmasc}{\textnormal{|}}
\newcommand{\Isep}{\textsuperscript{I}}
\newcommand{\luwmasc}{{\textsuperscript{𔖶}}}
\newcommand{\lbreak}{\hspace{2pt}\textnormal{||}\hspace{2pt}}
\newcommand{\ipa}[1]{\foreignlanguage{ipa}{#1}}
\newcommand{\pie}{\textsc{pie}}
\newcommand{\pac}{\textsc{pac}}
\newcommand{\parnumr}[2]{#1 \Roman{parcount}}
\newcommand{\parnuma}[2]{#1 \arabic{parcount}}
\newcounter{parcount}
\newenvironment{parnumbersa}[1][§]{%
   \par%
	 \everypar{\hangpara{3em}{1}\stepcounter{parcount} \textnormal{\normalsize\parnuma{#1}}
	 \tabto{2em}}%
}{}
\newenvironment{parnumbersr}[2][§]{%
   \par%
	 \everypar{\hangpara{6em}{1}\stepcounter{parcount} \textnormal{\normalsize\parnumr{#1}}
	 \tabto{5em}}%
}{}

\usepackage{csquotes}
\usepackage{fontspec}
\usepackage[main=brazil, bidi=basic]{babel}
\defaultfontfeatures{Renderer=Harfbuzz}

\babelfont[brazil]{rm}[
	SmallCapsFont=Gentium Plus,
	SmallCapsFeatures={Letters=SmallCaps}]{Crimson Pro}
\babelfont[brazil]{sf}{Noto Sans}
\babelfont[brazil]{tt}[Scale=0.8]{Mononoki Nerd Font}

\babelfont[german]{rm}[
	SmallCapsFont=Gentium Plus,
	SmallCapsFeatures={Letters=SmallCaps}]{Crimson Pro}
\babelfont[brazil]{sf}{Noto Sans}
\babelfont[brazil]{tt}[Scale=0.8]{Mononoki Nerd Font}

\babelfont[english]{rm}[
	SmallCapsFont=Gentium Plus,
	SmallCapsFeatures={Letters=SmallCaps}]{Crimson Pro}
\babelfont[english]{sf}{Noto Sans}
\babelfont[english]{tt}[Scale=0.8]{Mononoki Nerd Font}

\babelfont[german]{rm}[
	SmallCapsFont=Gentium Plus,
	SmallCapsFeatures={Letters=SmallCaps}]{Crimson Pro}
\babelfont[german]{sf}{Noto Sans}
\babelfont[german]{tt}[Scale=0.8]{Mononoki Nerd Font}


\babelprovide[import, onchar=ids fonts letters]{ancientgreek}
\babelfont[ancientgreek]{rm}{Brill}
\babeltags{grc = ancientgreek}

\babelprovide[import]{hebrew}
\babelfont[hebrew]{rm}[Scale=0.8]{Ezra SIL}

\babelprovide[onchar=ids fonts]{luwian}
\babelfont[luwian]{rm}[
	SmallCapsFont=Gentium Plus,
Script=Anatolian Hieroglyphs]{Noto Sans Anatolian Hieroglyphs}
\babelfont[luwian]{sf}[
	SmallCapsFont=Gentium Plus,
Script=Anatolian Hieroglyphs]{Noto Sans Anatolian Hieroglyphs}
\babelcharproperty{`𔐀}{locale}{luwian}
\babelcharproperty{`𔐁}{locale}{luwian}
\babelcharproperty{`𔐂}{locale}{luwian}
\babelcharproperty{`𔐃}{locale}{luwian}
\babelcharproperty{`𔐄}{locale}{luwian}
\babelcharproperty{`𔐅}{locale}{luwian}
\babelcharproperty{`𔐆}{locale}{luwian}
\babelcharproperty{`𔐇}{locale}{luwian}
\babelcharproperty{`𔐈}{locale}{luwian}
\babelcharproperty{`𔐉}{locale}{luwian}
\babelcharproperty{`𔐊}{locale}{luwian}
\babelcharproperty{`𔐋}{locale}{luwian}
\babelcharproperty{`𔐌}{locale}{luwian}
\babelcharproperty{`𔐍}{locale}{luwian}
\babelcharproperty{`𔐎}{locale}{luwian}
\babelcharproperty{`𔐏}{locale}{luwian}
\babelcharproperty{`𔐐}{locale}{luwian}
\babelcharproperty{`𔐑}{locale}{luwian}
\babelcharproperty{`𔐒}{locale}{luwian}
\babelcharproperty{`𔐓}{locale}{luwian}
\babelcharproperty{`𔐔}{locale}{luwian}
\babelcharproperty{`𔐕}{locale}{luwian}
\babelcharproperty{`𔐖}{locale}{luwian}
\babelcharproperty{`𔐗}{locale}{luwian}
\babelcharproperty{`𔐘}{locale}{luwian}
\babelcharproperty{`𔐙}{locale}{luwian}
\babelcharproperty{`𔐚}{locale}{luwian}
\babelcharproperty{`𔐛}{locale}{luwian}
\babelcharproperty{`𔐜}{locale}{luwian}
\babelcharproperty{`𔐝}{locale}{luwian}
\babelcharproperty{`𔐞}{locale}{luwian}
\babelcharproperty{`𔐟}{locale}{luwian}
\babelcharproperty{`𔐠}{locale}{luwian}
\babelcharproperty{`𔐡}{locale}{luwian}
\babelcharproperty{`𔐢}{locale}{luwian}
\babelcharproperty{`𔐣}{locale}{luwian}
\babelcharproperty{`𔐤}{locale}{luwian}
\babelcharproperty{`𔐥}{locale}{luwian}
\babelcharproperty{`𔐦}{locale}{luwian}
\babelcharproperty{`𔐧}{locale}{luwian}
\babelcharproperty{`𔐨}{locale}{luwian}
\babelcharproperty{`𔐩}{locale}{luwian}
\babelcharproperty{`𔐪}{locale}{luwian}
\babelcharproperty{`𔐫}{locale}{luwian}
\babelcharproperty{`𔐬}{locale}{luwian}
\babelcharproperty{`𔐭}{locale}{luwian}
\babelcharproperty{`𔐮}{locale}{luwian}
\babelcharproperty{`𔐯}{locale}{luwian}
\babelcharproperty{`𔐰}{locale}{luwian}
\babelcharproperty{`𔐱}{locale}{luwian}
\babelcharproperty{`𔐲}{locale}{luwian}
\babelcharproperty{`𔐳}{locale}{luwian}
\babelcharproperty{`𔐴}{locale}{luwian}
\babelcharproperty{`𔐵}{locale}{luwian}
\babelcharproperty{`𔐶}{locale}{luwian}
\babelcharproperty{`𔐷}{locale}{luwian}
\babelcharproperty{`𔐸}{locale}{luwian}
\babelcharproperty{`𔐹}{locale}{luwian}
\babelcharproperty{`𔐺}{locale}{luwian}
\babelcharproperty{`𔐻}{locale}{luwian}
\babelcharproperty{`𔐼}{locale}{luwian}
\babelcharproperty{`𔐽}{locale}{luwian}
\babelcharproperty{`𔐾}{locale}{luwian}
\babelcharproperty{`𔐿}{locale}{luwian}
\babelcharproperty{`𔑀}{locale}{luwian}
\babelcharproperty{`𔑁}{locale}{luwian}
\babelcharproperty{`𔑂}{locale}{luwian}
\babelcharproperty{`𔑃}{locale}{luwian}
\babelcharproperty{`𔑄}{locale}{luwian}
\babelcharproperty{`𔑅}{locale}{luwian}
\babelcharproperty{`𔑆}{locale}{luwian}
\babelcharproperty{`𔑇}{locale}{luwian}
\babelcharproperty{`𔑈}{locale}{luwian}
\babelcharproperty{`𔑉}{locale}{luwian}
\babelcharproperty{`𔑊}{locale}{luwian}
\babelcharproperty{`𔑋}{locale}{luwian}
\babelcharproperty{`𔑌}{locale}{luwian}
\babelcharproperty{`𔑍}{locale}{luwian}
\babelcharproperty{`𔑎}{locale}{luwian}
\babelcharproperty{`𔑏}{locale}{luwian}
\babelcharproperty{`𔑐}{locale}{luwian}
\babelcharproperty{`𔑑}{locale}{luwian}
\babelcharproperty{`𔑒}{locale}{luwian}
\babelcharproperty{`𔑓}{locale}{luwian}
\babelcharproperty{`𔑔}{locale}{luwian}
\babelcharproperty{`𔑕}{locale}{luwian}
\babelcharproperty{`𔑖}{locale}{luwian}
\babelcharproperty{`𔑗}{locale}{luwian}
\babelcharproperty{`𔑘}{locale}{luwian}
\babelcharproperty{`𔑙}{locale}{luwian}
\babelcharproperty{`𔑚}{locale}{luwian}
\babelcharproperty{`𔑛}{locale}{luwian}
\babelcharproperty{`𔑜}{locale}{luwian}
\babelcharproperty{`𔑝}{locale}{luwian}
\babelcharproperty{`𔑞}{locale}{luwian}
\babelcharproperty{`𔑟}{locale}{luwian}
\babelcharproperty{`𔑠}{locale}{luwian}
\babelcharproperty{`𔑡}{locale}{luwian}
\babelcharproperty{`𔑢}{locale}{luwian}
\babelcharproperty{`𔑣}{locale}{luwian}
\babelcharproperty{`𔑤}{locale}{luwian}
\babelcharproperty{`𔑥}{locale}{luwian}
\babelcharproperty{`𔑦}{locale}{luwian}
\babelcharproperty{`𔑧}{locale}{luwian}
\babelcharproperty{`𔑨}{locale}{luwian}
\babelcharproperty{`𔑩}{locale}{luwian}
\babelcharproperty{`𔑪}{locale}{luwian}
\babelcharproperty{`𔑫}{locale}{luwian}
\babelcharproperty{`𔑬}{locale}{luwian}
\babelcharproperty{`𔑭}{locale}{luwian}
\babelcharproperty{`𔑮}{locale}{luwian}
\babelcharproperty{`𔑯}{locale}{luwian}
\babelcharproperty{`𔑰}{locale}{luwian}
\babelcharproperty{`𔑱}{locale}{luwian}
\babelcharproperty{`𔑲}{locale}{luwian}
\babelcharproperty{`𔑳}{locale}{luwian}
\babelcharproperty{`𔑴}{locale}{luwian}
\babelcharproperty{`𔑵}{locale}{luwian}
\babelcharproperty{`𔑶}{locale}{luwian}
\babelcharproperty{`𔑷}{locale}{luwian}
\babelcharproperty{`𔑸}{locale}{luwian}
\babelcharproperty{`𔑹}{locale}{luwian}
\babelcharproperty{`𔑺}{locale}{luwian}
\babelcharproperty{`𔑻}{locale}{luwian}
\babelcharproperty{`𔑼}{locale}{luwian}
\babelcharproperty{`𔑽}{locale}{luwian}
\babelcharproperty{`𔑾}{locale}{luwian}
\babelcharproperty{`𔑿}{locale}{luwian}
\babelcharproperty{`𔒀}{locale}{luwian}
\babelcharproperty{`𔒁}{locale}{luwian}
\babelcharproperty{`𔒂}{locale}{luwian}
\babelcharproperty{`𔒃}{locale}{luwian}
\babelcharproperty{`𔒄}{locale}{luwian}
\babelcharproperty{`𔒅}{locale}{luwian}
\babelcharproperty{`𔒆}{locale}{luwian}
\babelcharproperty{`𔒇}{locale}{luwian}
\babelcharproperty{`𔒈}{locale}{luwian}
\babelcharproperty{`𔒉}{locale}{luwian}
\babelcharproperty{`𔒊}{locale}{luwian}
\babelcharproperty{`𔒋}{locale}{luwian}
\babelcharproperty{`𔒌}{locale}{luwian}
\babelcharproperty{`𔒍}{locale}{luwian}
\babelcharproperty{`𔒎}{locale}{luwian}
\babelcharproperty{`𔒏}{locale}{luwian}
\babelcharproperty{`𔒐}{locale}{luwian}
\babelcharproperty{`𔒑}{locale}{luwian}
\babelcharproperty{`𔒒}{locale}{luwian}
\babelcharproperty{`𔒓}{locale}{luwian}
\babelcharproperty{`𔒔}{locale}{luwian}
\babelcharproperty{`𔒕}{locale}{luwian}
\babelcharproperty{`𔒖}{locale}{luwian}
\babelcharproperty{`𔒗}{locale}{luwian}
\babelcharproperty{`𔒘}{locale}{luwian}
\babelcharproperty{`𔒙}{locale}{luwian}
\babelcharproperty{`𔒚}{locale}{luwian}
\babelcharproperty{`𔒛}{locale}{luwian}
\babelcharproperty{`𔒜}{locale}{luwian}
\babelcharproperty{`𔒝}{locale}{luwian}
\babelcharproperty{`𔒞}{locale}{luwian}
\babelcharproperty{`𔒟}{locale}{luwian}
\babelcharproperty{`𔒠}{locale}{luwian}
\babelcharproperty{`𔒡}{locale}{luwian}
\babelcharproperty{`𔒢}{locale}{luwian}
\babelcharproperty{`𔒣}{locale}{luwian}
\babelcharproperty{`𔒤}{locale}{luwian}
\babelcharproperty{`𔒥}{locale}{luwian}
\babelcharproperty{`𔒦}{locale}{luwian}
\babelcharproperty{`𔒧}{locale}{luwian}
\babelcharproperty{`𔒨}{locale}{luwian}
\babelcharproperty{`𔒩}{locale}{luwian}
\babelcharproperty{`𔒪}{locale}{luwian}
\babelcharproperty{`𔒫}{locale}{luwian}
\babelcharproperty{`𔒬}{locale}{luwian}
\babelcharproperty{`𔒭}{locale}{luwian}
\babelcharproperty{`𔒮}{locale}{luwian}
\babelcharproperty{`𔒯}{locale}{luwian}
\babelcharproperty{`𔒰}{locale}{luwian}
\babelcharproperty{`𔒱}{locale}{luwian}
\babelcharproperty{`𔒲}{locale}{luwian}
\babelcharproperty{`𔒳}{locale}{luwian}
\babelcharproperty{`𔒴}{locale}{luwian}
\babelcharproperty{`𔒵}{locale}{luwian}
\babelcharproperty{`𔒶}{locale}{luwian}
\babelcharproperty{`𔒷}{locale}{luwian}
\babelcharproperty{`𔒸}{locale}{luwian}
\babelcharproperty{`𔒹}{locale}{luwian}
\babelcharproperty{`𔒺}{locale}{luwian}
\babelcharproperty{`𔒻}{locale}{luwian}
\babelcharproperty{`𔒼}{locale}{luwian}
\babelcharproperty{`𔒽}{locale}{luwian}
\babelcharproperty{`𔒾}{locale}{luwian}
\babelcharproperty{`𔒿}{locale}{luwian}
\babelcharproperty{`𔓀}{locale}{luwian}
\babelcharproperty{`𔓁}{locale}{luwian}
\babelcharproperty{`𔓂}{locale}{luwian}
\babelcharproperty{`𔓃}{locale}{luwian}
\babelcharproperty{`𔓄}{locale}{luwian}
\babelcharproperty{`𔓅}{locale}{luwian}
\babelcharproperty{`𔓆}{locale}{luwian}
\babelcharproperty{`𔓇}{locale}{luwian}
\babelcharproperty{`𔓈}{locale}{luwian}
\babelcharproperty{`𔓉}{locale}{luwian}
\babelcharproperty{`𔓊}{locale}{luwian}
\babelcharproperty{`𔓋}{locale}{luwian}
\babelcharproperty{`𔓌}{locale}{luwian}
\babelcharproperty{`𔓍}{locale}{luwian}
\babelcharproperty{`𔓎}{locale}{luwian}
\babelcharproperty{`𔓏}{locale}{luwian}
\babelcharproperty{`𔓐}{locale}{luwian}
\babelcharproperty{`𔓑}{locale}{luwian}
\babelcharproperty{`𔓒}{locale}{luwian}
\babelcharproperty{`𔓓}{locale}{luwian}
\babelcharproperty{`𔓔}{locale}{luwian}
\babelcharproperty{`𔓕}{locale}{luwian}
\babelcharproperty{`𔓖}{locale}{luwian}
\babelcharproperty{`𔓗}{locale}{luwian}
\babelcharproperty{`𔓘}{locale}{luwian}
\babelcharproperty{`𔓙}{locale}{luwian}
\babelcharproperty{`𔓚}{locale}{luwian}
\babelcharproperty{`𔓛}{locale}{luwian}
\babelcharproperty{`𔓜}{locale}{luwian}
\babelcharproperty{`𔓝}{locale}{luwian}
\babelcharproperty{`𔓞}{locale}{luwian}
\babelcharproperty{`𔓟}{locale}{luwian}
\babelcharproperty{`𔓠}{locale}{luwian}
\babelcharproperty{`𔓡}{locale}{luwian}
\babelcharproperty{`𔓢}{locale}{luwian}
\babelcharproperty{`𔓣}{locale}{luwian}
\babelcharproperty{`𔓤}{locale}{luwian}
\babelcharproperty{`𔓥}{locale}{luwian}
\babelcharproperty{`𔓦}{locale}{luwian}
\babelcharproperty{`𔓧}{locale}{luwian}
\babelcharproperty{`𔓨}{locale}{luwian}
\babelcharproperty{`𔓩}{locale}{luwian}
\babelcharproperty{`𔓪}{locale}{luwian}
\babelcharproperty{`𔓫}{locale}{luwian}
\babelcharproperty{`𔓬}{locale}{luwian}
\babelcharproperty{`𔓭}{locale}{luwian}
\babelcharproperty{`𔓮}{locale}{luwian}
\babelcharproperty{`𔓯}{locale}{luwian}
\babelcharproperty{`𔓰}{locale}{luwian}
\babelcharproperty{`𔓱}{locale}{luwian}
\babelcharproperty{`𔓲}{locale}{luwian}
\babelcharproperty{`𔓳}{locale}{luwian}
\babelcharproperty{`𔓴}{locale}{luwian}
\babelcharproperty{`𔓵}{locale}{luwian}
\babelcharproperty{`𔓶}{locale}{luwian}
\babelcharproperty{`𔓷}{locale}{luwian}
\babelcharproperty{`𔓸}{locale}{luwian}
\babelcharproperty{`𔓹}{locale}{luwian}
\babelcharproperty{`𔓺}{locale}{luwian}
\babelcharproperty{`𔓻}{locale}{luwian}
\babelcharproperty{`𔓼}{locale}{luwian}
\babelcharproperty{`𔓽}{locale}{luwian}
\babelcharproperty{`𔓾}{locale}{luwian}
\babelcharproperty{`𔓿}{locale}{luwian}
\babelcharproperty{`𔔀}{locale}{luwian}
\babelcharproperty{`𔔁}{locale}{luwian}
\babelcharproperty{`𔔂}{locale}{luwian}
\babelcharproperty{`𔔃}{locale}{luwian}
\babelcharproperty{`𔔄}{locale}{luwian}
\babelcharproperty{`𔔅}{locale}{luwian}
\babelcharproperty{`𔔆}{locale}{luwian}
\babelcharproperty{`𔔇}{locale}{luwian}
\babelcharproperty{`𔔈}{locale}{luwian}
\babelcharproperty{`𔔉}{locale}{luwian}
\babelcharproperty{`𔔊}{locale}{luwian}
\babelcharproperty{`𔔋}{locale}{luwian}
\babelcharproperty{`𔔌}{locale}{luwian}
\babelcharproperty{`𔔍}{locale}{luwian}
\babelcharproperty{`𔔎}{locale}{luwian}
\babelcharproperty{`𔔏}{locale}{luwian}
\babelcharproperty{`𔔐}{locale}{luwian}
\babelcharproperty{`𔔑}{locale}{luwian}
\babelcharproperty{`𔔒}{locale}{luwian}
\babelcharproperty{`𔔓}{locale}{luwian}
\babelcharproperty{`𔔔}{locale}{luwian}
\babelcharproperty{`𔔕}{locale}{luwian}
\babelcharproperty{`𔔖}{locale}{luwian}
\babelcharproperty{`𔔗}{locale}{luwian}
\babelcharproperty{`𔔘}{locale}{luwian}
\babelcharproperty{`𔔙}{locale}{luwian}
\babelcharproperty{`𔔚}{locale}{luwian}
\babelcharproperty{`𔔛}{locale}{luwian}
\babelcharproperty{`𔔜}{locale}{luwian}
\babelcharproperty{`𔔝}{locale}{luwian}
\babelcharproperty{`𔔞}{locale}{luwian}
\babelcharproperty{`𔔟}{locale}{luwian}
\babelcharproperty{`𔔠}{locale}{luwian}
\babelcharproperty{`𔔡}{locale}{luwian}
\babelcharproperty{`𔔢}{locale}{luwian}
\babelcharproperty{`𔔣}{locale}{luwian}
\babelcharproperty{`𔔤}{locale}{luwian}
\babelcharproperty{`𔔥}{locale}{luwian}
\babelcharproperty{`𔔦}{locale}{luwian}
\babelcharproperty{`𔔧}{locale}{luwian}
\babelcharproperty{`𔔨}{locale}{luwian}
\babelcharproperty{`𔔩}{locale}{luwian}
\babelcharproperty{`𔔪}{locale}{luwian}
\babelcharproperty{`𔔫}{locale}{luwian}
\babelcharproperty{`𔔬}{locale}{luwian}
\babelcharproperty{`𔔭}{locale}{luwian}
\babelcharproperty{`𔔮}{locale}{luwian}
\babelcharproperty{`𔔯}{locale}{luwian}
\babelcharproperty{`𔔰}{locale}{luwian}
\babelcharproperty{`𔔱}{locale}{luwian}
\babelcharproperty{`𔔲}{locale}{luwian}
\babelcharproperty{`𔔳}{locale}{luwian}
\babelcharproperty{`𔔴}{locale}{luwian}
\babelcharproperty{`𔔵}{locale}{luwian}
\babelcharproperty{`𔔶}{locale}{luwian}
\babelcharproperty{`𔔷}{locale}{luwian}
\babelcharproperty{`𔔸}{locale}{luwian}
\babelcharproperty{`𔔹}{locale}{luwian}
\babelcharproperty{`𔔺}{locale}{luwian}
\babelcharproperty{`𔔻}{locale}{luwian}
\babelcharproperty{`𔔼}{locale}{luwian}
\babelcharproperty{`𔔽}{locale}{luwian}
\babelcharproperty{`𔔾}{locale}{luwian}
\babelcharproperty{`𔔿}{locale}{luwian}
\babelcharproperty{`𔕀}{locale}{luwian}
\babelcharproperty{`𔕁}{locale}{luwian}
\babelcharproperty{`𔕂}{locale}{luwian}
\babelcharproperty{`𔕃}{locale}{luwian}
\babelcharproperty{`𔕄}{locale}{luwian}
\babelcharproperty{`𔕅}{locale}{luwian}
\babelcharproperty{`𔕆}{locale}{luwian}
\babelcharproperty{`𔕇}{locale}{luwian}
\babelcharproperty{`𔕈}{locale}{luwian}
\babelcharproperty{`𔕉}{locale}{luwian}
\babelcharproperty{`𔕊}{locale}{luwian}
\babelcharproperty{`𔕋}{locale}{luwian}
\babelcharproperty{`𔕌}{locale}{luwian}
\babelcharproperty{`𔕍}{locale}{luwian}
\babelcharproperty{`𔕎}{locale}{luwian}
\babelcharproperty{`𔕏}{locale}{luwian}
\babelcharproperty{`𔕐}{locale}{luwian}
\babelcharproperty{`𔕑}{locale}{luwian}
\babelcharproperty{`𔕒}{locale}{luwian}
\babelcharproperty{`𔕓}{locale}{luwian}
\babelcharproperty{`𔕔}{locale}{luwian}
\babelcharproperty{`𔕕}{locale}{luwian}
\babelcharproperty{`𔕖}{locale}{luwian}
\babelcharproperty{`𔕗}{locale}{luwian}
\babelcharproperty{`𔕘}{locale}{luwian}
\babelcharproperty{`𔕙}{locale}{luwian}
\babelcharproperty{`𔕚}{locale}{luwian}
\babelcharproperty{`𔕛}{locale}{luwian}
\babelcharproperty{`𔕜}{locale}{luwian}
\babelcharproperty{`𔕝}{locale}{luwian}
\babelcharproperty{`𔕞}{locale}{luwian}
\babelcharproperty{`𔕟}{locale}{luwian}
\babelcharproperty{`𔕠}{locale}{luwian}
\babelcharproperty{`𔕡}{locale}{luwian}
\babelcharproperty{`𔕢}{locale}{luwian}
\babelcharproperty{`𔕣}{locale}{luwian}
\babelcharproperty{`𔕤}{locale}{luwian}
\babelcharproperty{`𔕥}{locale}{luwian}
\babelcharproperty{`𔕦}{locale}{luwian}
\babelcharproperty{`𔕧}{locale}{luwian}
\babelcharproperty{`𔕨}{locale}{luwian}
\babelcharproperty{`𔕩}{locale}{luwian}
\babelcharproperty{`𔕪}{locale}{luwian}
\babelcharproperty{`𔕫}{locale}{luwian}
\babelcharproperty{`𔕬}{locale}{luwian}
\babelcharproperty{`𔕭}{locale}{luwian}
\babelcharproperty{`𔕮}{locale}{luwian}
\babelcharproperty{`𔕯}{locale}{luwian}
\babelcharproperty{`𔕰}{locale}{luwian}
\babelcharproperty{`𔕱}{locale}{luwian}
\babelcharproperty{`𔕲}{locale}{luwian}
\babelcharproperty{`𔕳}{locale}{luwian}
\babelcharproperty{`𔕴}{locale}{luwian}
\babelcharproperty{`𔕵}{locale}{luwian}
\babelcharproperty{`𔕶}{locale}{luwian}
\babelcharproperty{`𔕷}{locale}{luwian}
\babelcharproperty{`𔕸}{locale}{luwian}
\babelcharproperty{`𔕹}{locale}{luwian}
\babelcharproperty{`𔕺}{locale}{luwian}
\babelcharproperty{`𔕻}{locale}{luwian}
\babelcharproperty{`𔕼}{locale}{luwian}
\babelcharproperty{`𔕽}{locale}{luwian}
\babelcharproperty{`𔕾}{locale}{luwian}
\babelcharproperty{`𔕿}{locale}{luwian}
\babelcharproperty{`𔖀}{locale}{luwian}
\babelcharproperty{`𔖁}{locale}{luwian}
\babelcharproperty{`𔖂}{locale}{luwian}
\babelcharproperty{`𔖃}{locale}{luwian}
\babelcharproperty{`𔖄}{locale}{luwian}
\babelcharproperty{`𔖅}{locale}{luwian}
\babelcharproperty{`𔖆}{locale}{luwian}
\babelcharproperty{`𔖇}{locale}{luwian}
\babelcharproperty{`𔖈}{locale}{luwian}
\babelcharproperty{`𔖉}{locale}{luwian}
\babelcharproperty{`𔖊}{locale}{luwian}
\babelcharproperty{`𔖋}{locale}{luwian}
\babelcharproperty{`𔖌}{locale}{luwian}
\babelcharproperty{`𔖍}{locale}{luwian}
\babelcharproperty{`𔖎}{locale}{luwian}
\babelcharproperty{`𔖏}{locale}{luwian}
\babelcharproperty{`𔖐}{locale}{luwian}
\babelcharproperty{`𔖑}{locale}{luwian}
\babelcharproperty{`𔖒}{locale}{luwian}
\babelcharproperty{`𔖓}{locale}{luwian}
\babelcharproperty{`𔖔}{locale}{luwian}
\babelcharproperty{`𔖕}{locale}{luwian}
\babelcharproperty{`𔖖}{locale}{luwian}
\babelcharproperty{`𔖗}{locale}{luwian}
\babelcharproperty{`𔖘}{locale}{luwian}
\babelcharproperty{`𔖙}{locale}{luwian}
\babelcharproperty{`𔖚}{locale}{luwian}
\babelcharproperty{`𔖛}{locale}{luwian}
\babelcharproperty{`𔖜}{locale}{luwian}
\babelcharproperty{`𔖝}{locale}{luwian}
\babelcharproperty{`𔖞}{locale}{luwian}
\babelcharproperty{`𔖟}{locale}{luwian}
\babelcharproperty{`𔖠}{locale}{luwian}
\babelcharproperty{`𔖡}{locale}{luwian}
\babelcharproperty{`𔖢}{locale}{luwian}
\babelcharproperty{`𔖣}{locale}{luwian}
\babelcharproperty{`𔖤}{locale}{luwian}
\babelcharproperty{`𔖥}{locale}{luwian}
\babelcharproperty{`𔖦}{locale}{luwian}
\babelcharproperty{`𔖧}{locale}{luwian}
\babelcharproperty{`𔖨}{locale}{luwian}
\babelcharproperty{`𔖩}{locale}{luwian}
\babelcharproperty{`𔖪}{locale}{luwian}
\babelcharproperty{`𔖫}{locale}{luwian}
\babelcharproperty{`𔖬}{locale}{luwian}
\babelcharproperty{`𔖭}{locale}{luwian}
\babelcharproperty{`𔖮}{locale}{luwian}
\babelcharproperty{`𔖯}{locale}{luwian}
\babelcharproperty{`𔖰}{locale}{luwian}
\babelcharproperty{`𔖱}{locale}{luwian}
\babelcharproperty{`𔖲}{locale}{luwian}
\babelcharproperty{`𔖳}{locale}{luwian}
\babelcharproperty{`𔖴}{locale}{luwian}
\babelcharproperty{`𔖵}{locale}{luwian}
\babelcharproperty{`𔖶}{locale}{luwian}
\babelcharproperty{`𔖷}{locale}{luwian}
\babelcharproperty{`𔖸}{locale}{luwian}
\babelcharproperty{`𔖹}{locale}{luwian}
\babelcharproperty{`𔖺}{locale}{luwian}
\babelcharproperty{`𔖻}{locale}{luwian}
\babelcharproperty{`𔖼}{locale}{luwian}
\babelcharproperty{`𔖽}{locale}{luwian}
\babelcharproperty{`𔖾}{locale}{luwian}
\babelcharproperty{`𔖿}{locale}{luwian}
\babelcharproperty{`𔗀}{locale}{luwian}
\babelcharproperty{`𔗁}{locale}{luwian}
\babelcharproperty{`𔗂}{locale}{luwian}
\babelcharproperty{`𔗃}{locale}{luwian}
\babelcharproperty{`𔗄}{locale}{luwian}
\babelcharproperty{`𔗅}{locale}{luwian}
\babelcharproperty{`𔗆}{locale}{luwian}
\babelcharproperty{`𔗇}{locale}{luwian}
\babelcharproperty{`𔗈}{locale}{luwian}
\babelcharproperty{`𔗉}{locale}{luwian}
\babelcharproperty{`𔗊}{locale}{luwian}
\babelcharproperty{`𔗋}{locale}{luwian}
\babelcharproperty{`𔗌}{locale}{luwian}
\babelcharproperty{`𔗍}{locale}{luwian}
\babelcharproperty{`𔗎}{locale}{luwian}
\babelcharproperty{`𔗏}{locale}{luwian}
\babelcharproperty{`𔗐}{locale}{luwian}
\babelcharproperty{`𔗑}{locale}{luwian}
\babelcharproperty{`𔗒}{locale}{luwian}
\babelcharproperty{`𔗓}{locale}{luwian}
\babelcharproperty{`𔗔}{locale}{luwian}
\babelcharproperty{`𔗕}{locale}{luwian}
\babelcharproperty{`𔗖}{locale}{luwian}
\babelcharproperty{`𔗗}{locale}{luwian}
\babelcharproperty{`𔗘}{locale}{luwian}
\babelcharproperty{`𔗙}{locale}{luwian}
\babelcharproperty{`𔗚}{locale}{luwian}
\babelcharproperty{`𔗛}{locale}{luwian}
\babelcharproperty{`𔗜}{locale}{luwian}
\babelcharproperty{`𔗝}{locale}{luwian}
\babelcharproperty{`𔗞}{locale}{luwian}
\babelcharproperty{`𔗟}{locale}{luwian}
\babelcharproperty{`𔗠}{locale}{luwian}
\babelcharproperty{`𔗡}{locale}{luwian}
\babelcharproperty{`𔗢}{locale}{luwian}
\babelcharproperty{`𔗣}{locale}{luwian}
\babelcharproperty{`𔗤}{locale}{luwian}
\babelcharproperty{`𔗥}{locale}{luwian}
\babelcharproperty{`𔗦}{locale}{luwian}
\babelcharproperty{`𔗧}{locale}{luwian}
\babelcharproperty{`𔗨}{locale}{luwian}
\babelcharproperty{`𔗩}{locale}{luwian}
\babelcharproperty{`𔗪}{locale}{luwian}
\babelcharproperty{`𔗫}{locale}{luwian}
\babelcharproperty{`𔗬}{locale}{luwian}
\babelcharproperty{`𔗭}{locale}{luwian}
\babelcharproperty{`𔗮}{locale}{luwian}
\babelcharproperty{`𔗯}{locale}{luwian}
\babelcharproperty{`𔗰}{locale}{luwian}
\babelcharproperty{`𔗱}{locale}{luwian}
\babelcharproperty{`𔗲}{locale}{luwian}
\babelcharproperty{`𔗳}{locale}{luwian}
\babelcharproperty{`𔗴}{locale}{luwian}
\babelcharproperty{`𔗵}{locale}{luwian}
\babelcharproperty{`𔗶}{locale}{luwian}
\babelcharproperty{`𔗷}{locale}{luwian}
\babelcharproperty{`𔗸}{locale}{luwian}
\babelcharproperty{`𔗹}{locale}{luwian}
\babelcharproperty{`𔗺}{locale}{luwian}
\babelcharproperty{`𔗻}{locale}{luwian}
\babelcharproperty{`𔗼}{locale}{luwian}
\babelcharproperty{`𔗽}{locale}{luwian}
\babelcharproperty{`𔗾}{locale}{luwian}
\babelcharproperty{`𔗿}{locale}{luwian}
\babelcharproperty{`𔘀}{locale}{luwian}
\babelcharproperty{`𔘁}{locale}{luwian}
\babelcharproperty{`𔘂}{locale}{luwian}
\babelcharproperty{`𔘃}{locale}{luwian}
\babelcharproperty{`𔘄}{locale}{luwian}
\babelcharproperty{`𔘅}{locale}{luwian}
\babelcharproperty{`𔘆}{locale}{luwian}
\babelcharproperty{`𔘇}{locale}{luwian}
\babelcharproperty{`𔘈}{locale}{luwian}
\babelcharproperty{`𔘉}{locale}{luwian}
\babelcharproperty{`𔘊}{locale}{luwian}
\babelcharproperty{`𔘋}{locale}{luwian}
\babelcharproperty{`𔘌}{locale}{luwian}
\babelcharproperty{`𔘍}{locale}{luwian}
\babelcharproperty{`𔘎}{locale}{luwian}
\babelcharproperty{`𔘏}{locale}{luwian}
\babelcharproperty{`𔘐}{locale}{luwian}
\babelcharproperty{`𔘑}{locale}{luwian}
\babelcharproperty{`𔘒}{locale}{luwian}
\babelcharproperty{`𔘓}{locale}{luwian}
\babelcharproperty{`𔘔}{locale}{luwian}
\babelcharproperty{`𔘕}{locale}{luwian}
\babelcharproperty{`𔘖}{locale}{luwian}
\babelcharproperty{`𔘗}{locale}{luwian}
\babelcharproperty{`𔘘}{locale}{luwian}
\babelcharproperty{`𔘙}{locale}{luwian}
\babelcharproperty{`𔘚}{locale}{luwian}
\babelcharproperty{`𔘛}{locale}{luwian}
\babelcharproperty{`𔘜}{locale}{luwian}
\babelcharproperty{`𔘝}{locale}{luwian}
\babelcharproperty{`𔘞}{locale}{luwian}
\babelcharproperty{`𔘟}{locale}{luwian}
\babelcharproperty{`𔘠}{locale}{luwian}
\babelcharproperty{`𔘡}{locale}{luwian}
\babelcharproperty{`𔘢}{locale}{luwian}
\babelcharproperty{`𔘣}{locale}{luwian}
\babelcharproperty{`𔘤}{locale}{luwian}
\babelcharproperty{`𔘥}{locale}{luwian}
\babelcharproperty{`𔘦}{locale}{luwian}
\babelcharproperty{`𔘧}{locale}{luwian}
\babelcharproperty{`𔘨}{locale}{luwian}
\babelcharproperty{`𔘩}{locale}{luwian}
\babelcharproperty{`𔘪}{locale}{luwian}
\babelcharproperty{`𔘫}{locale}{luwian}
\babelcharproperty{`𔘬}{locale}{luwian}
\babelcharproperty{`𔘭}{locale}{luwian}
\babelcharproperty{`𔘮}{locale}{luwian}
\babelcharproperty{`𔘯}{locale}{luwian}
\babelcharproperty{`𔘰}{locale}{luwian}
\babelcharproperty{`𔘱}{locale}{luwian}
\babelcharproperty{`𔘲}{locale}{luwian}
\babelcharproperty{`𔘳}{locale}{luwian}
\babelcharproperty{`𔘴}{locale}{luwian}
\babelcharproperty{`𔘵}{locale}{luwian}
\babelcharproperty{`𔘶}{locale}{luwian}
\babelcharproperty{`𔘷}{locale}{luwian}
\babelcharproperty{`𔘸}{locale}{luwian}
\babelcharproperty{`𔘹}{locale}{luwian}
\babelcharproperty{`𔘺}{locale}{luwian}
\babelcharproperty{`𔘻}{locale}{luwian}
\babelcharproperty{`𔘼}{locale}{luwian}
\babelcharproperty{`𔘽}{locale}{luwian}
\babelcharproperty{`𔘾}{locale}{luwian}
\babelcharproperty{`𔘿}{locale}{luwian}
\babelcharproperty{`𔙀}{locale}{luwian}
\babelcharproperty{`𔙁}{locale}{luwian}
\babelcharproperty{`𔙂}{locale}{luwian}
\babelcharproperty{`𔙃}{locale}{luwian}
\babelcharproperty{`𔙄}{locale}{luwian}
\babelcharproperty{`𔙅}{locale}{luwian}
\babelcharproperty{`𔙆}{locale}{luwian}


\babelprovide{hittite}
\babelfont[hittite]{rm}{UllikummiA}


\babelprovide{ipa}
\babelfont[ipa]{rm}{Gentium Plus}

\babelprovide[import,onchar=ids fonts]{sanskrit}
\babelfont[sanskrit]{rm}[Scale=1]{Noto Serif Devanagari}
\babelfont[sanskrit]{sf}[Scale=1]{Noto Sans Devanagari}


\usepackage{hyphenat}
% TeX root=../main.tex

\hyphenation{
	hi-e-ro-glí-fi-co
	HAL-PA
	Tar-hun-ta
	tex-to
	hie-ro-gly-phen
	lu-wi-schen
	Bo-ğaz-köy
	Lu-wian
	cha-ma-da
}


\title{Luvita Hieroglífico: Aula 5}
\author{Caio Geraldes}
\date{3 de setembro de 2024}

\usepackage[backend=biber,
	style=abnt,
	repeatfields,
	scbib,
	ittitles,
	indent,
	giveninits,
	justify,
	noslsn,
	natbib,
	extrayear,
]{biblatex}
\addbibresource{../../../Bibliografia/biblio.bib}
\DeclareCiteCommand*{\citeabbrev}%your new citecommand \citetitle*
{\boolfalse{citetracker}%
	\boolfalse{pagetracker}%
	\usebibmacro{prenote}}
{\ifciteindex%
	{\indexfield{indextitle}}
	{}%
	\printtext[bibhyperref]{\printfield{shorttitle}}}%like \citetile, 
%only added \printtext[bibhyperref]{...} in this line
{\multicitedelim}
{}

\newcommand{\GrHL}[1]{\citeabbrev*{Hoffner2008} #1}




\usepackage{tikz} %for all basic options
\usepackage{tikz-qtree} %for simple tree syntax
\usepgflibrary{arrows} %for arrow endings
\usetikzlibrary{positioning,shapes.multipart} %for structured nodes
\usetikzlibrary{tikzmark}
\usepackage{tree-dvips}

\usepackage{multirow}

\begin{document}

\setlength{\Exlabelsep}{0.5em}
\setlength{\SubExleftmargin}{1.5em}

\frontmatter

\mainmatter%

\maketitle


% TeX root=../main.tex

\chapter{Leitura: \logo{KARATEPE}}

\clearpage

\begin{parnumbersr}

	\raggedright%
	\itshape%

	\lmasc{}\logo{EGO}-mi
	\spac{}\logo{(LITUUS)}á-za-ti-i-wa/i-da-sá
	\logo{(DEUS)}\logo{SOL}-mi-sá
	\logo{(CAPUT)}-ti-i-sá
	\logo{(DEUS)}\logo{TONITRUS}-hu-ta-sa
	\logo{SERVUS}-la/i-sá

	á-wa/i+ra/i-ku-sa-wa/i
	\lbreak{}
	\logo{REL}-i-na
	\logo{MAGNUS}+ra/i-nu-wa-ta
	á-TANA-wa/i-ní-i-sá\logo{(URBS)}
	\logo{REX}-ti-sá

	wa/i-mu-u
	\logo{(DEUS).TONITRUS}-hu-za-sa á-TANA-wa/i-\emph{||}-ia\logo{(URBS)}
	\logo{MATER}-na-tí-na tá-ti-ha i-zi-i-da

	\lmasc{}\logo{ARHA}-ha-wa/i
	\lmasc{}la+ra/i+a-nú-ha
	\lmasc{}á-TANA-wa/i-na\logo{(URBS)}

	\lmasc{}\logo{(“MANUS”)}la-tara/i-ha-ha-wá/í
	\lmasc{}á-TANA-wá/í-za\logo{(URBS)} \lmasc{}\logo{“TERRA+X”}{(-)}wá/í+ra/i-za
	\lmasc{}zi-na \lmasc{}\logo{(“OCCIDENS”)}i-pa-mi \lmasc{}\logo{VERSUS}-ia-na
	\lmasc{}zi-pa-wá/í \logo{(ORIENS)}ki-sà-ta-mi-i \lmasc{}VERSUS-na

	\lmasc{}á-mi-ia-za-há-wa/i \logo{(“DIES<”>)}ha-lí-za
	\lmasc{}á-TANA-wá/í-ia\logo{(URBS)} \lmasc{}\logo{OMNIS+}MI-ma
	\logo{(“BONUS”)}sa-na-wa/i-ia \lmasc{}\logo{(“CORNU+RA/I”)}su+ra/i-sa
	\lmasc{}\logo{(LINGERE)}ha-sa-sa-ha á-sá-ta


	\lmasc{}\logo{(“MANUS”)}su-wá/í-ha-ha-wá/í
	\lmasc{}pa-há+ra/i-wa/i-ní-zi\logo{(URBS)}
	\lmasc{}\logo{(<“>*255”)}ka-ru-na-zi

	\lmasc{}\logo{EQUUS.ANIMA}-zú-ha-wa/i-ta \logo{(EQUUS.ANIMA)}á-zú-wa/i
	\lmasc{}\logo{SUPER}+ra/i-ta \lmasc{}i-zi-i-ha

	\logo{EXERCITUS}-lu/a/i-za-pa-wa/i-ta \lmasc{}\logo{EXERCITUS}-lu/a/i-ní
	\lmasc{}\logo{SUPER}+ra/i-ta \lmasc{}i-zi-i-há

	\lmasc{}\logo{(<“>SCUTUM”)}hara/i-li-pa-wa/i-ta
	\lmasc{}\logo{(“SCUTUM”)}hara/i-li \lmasc{}\logo{SUPER}+ra/i-ta
	\lmasc{}i-zi-i-há $[$\logo{OMNIS-}MI-ma-za
					\lmasc{}\logo{{(DEUS)}TONITRUS}-hu-ta-tí \logo{DEUS}-na-ri+i-ha$]$


\end{parnumbersr}


\vspace{10pt}
\hrule
\vspace{10pt}


\setcounter{parcount}{0}
\begin{parnumbersr}

	\raggedright%
	\itshape%

	amu=mi Azatiwadas tiwadamis \logo{CAPUT}-tis Tarhunt{(a)}s hudarlis,

	Awarikus=wa \lbreak{} kwin uranuwata Adanawanis hantawatis,

	*a=wa=mu Tarhunz Adanawaya anatin tadin=ha izida.

	arha=ha=wa laranuha Adanawan.

	lataraha=ha=wa Adanawan=za walirin=za zin ipami tawiyan zin=pa=wa kistami
	tawiyan.

	amiyanza=ha=wa halinza Adanawaya tanima sanawiya \logo{(“CORNU+RA/I”)}-suras
	hasas=ha asta.

	suwaha=ha=wa Paharawaninzi karunanzi,

	azun=ha=wa=ta azuwi sara iziha,

	kulanin=za=pa=wa kulani sara iziha,

	haralin=pa=wa=ta harali sara iziha, taniman=za Tarhuntadi masanari=ha.


\end{parnumbersr}


\clearpage%
\noindent\textbf{Tradução}

[I] Eu sou Azatiwada, homem abençoado{(?)} pelo sol, servo de Tarhunta, [II] que
Awariku, rei de Adanawa, elevou, [III] e Tarhunta me fez da (cidade de)
Adanawa mãe e pai. [IV] Eu fiz (a cidade de) Adanawa prosperar, [V] eu estendi
a planície de Adanawa de um lado em direção ao ocidente, do outro em direção
ao oriente [VI] e, nos meus dias, havia em Adanawa todos os bens, abundância e
saciedade (\emph{ou} luxo). [VII] Eu enchi os silos de Pahara [VIII] e fiz
cavalo e mais cavalo, [IX] e fiz exército e mais exército, [X] e fiz escudo e
mais escudo, tudo por {(graça de?)} Tarhunta e pelos deuses (\emph{ou} pela
graça dos deuses).

\bigskip
\noindent\textbf{Notas}

\smallskip
\noindent\textbf{§I}\tabto{2em}
\textbf{\emph{tiwadamis}} `abençoado/a pelo deus Sol': o nome do deus Sol em
luvita é \emph{Tiwad{(a)}-}\footnote{Ver formas quase completas em KÜRTÜL, §6 e
	KARKAMIŠ A15\emph{b}, §1.} e esta forma utiliza o sufixo de formação de
adjetivos \emph{-ami-}.
O sentido específico deo adjetivo como \emph{abençoado/a} é gerado a partir do
fenício \emph{h-brk}.
\textbf{CAPUT-\emph{tis}} `pessoa, homem': a forma subjacente é incerta, nunca
sendo escrita em sua completude fonologicamente.
O termo \emph{ziti-} `homem' parece apenas ocorrer com L.313 𔕠 VIR, fazendo-nos
crer que L.10 𔐉 CAPUT é reservada para outro elemento semântico.
No entanto, em diversas passagens de KARATEPE, CAPUT\emph{-ti-} corresponde ao
fenício \emph{ʾdm} `homem'.
\textbf{\emph{hudarlis}} `servo': fonologia reconstruída a partir do
luvita cuneiforme \emph{hudarli-}.

\smallskip
\noindent\textbf{§II}\tabto{2em}
\textbf{\emph{Awarikus=wa kwin}}: oração relativa com o sujeito antecedendo o
pronome que recupera \emph{amu} `eu' de §1. Awariku foi por vezes
identificado com o rei Urikki de Que, tributário de Tiglate-pileser III\@,
mas a evidência é pouca e há a possibilidade de ser o avô deste.

\smallskip
\noindent\textbf{§III}\tabto{2em}
\textbf{\emph{Tarhunz}} `Tarhunta': a divindade Tarhunta no texto fenício é
traduzida como \emph{bʿl} `Baal \slash{} senhor'.
A variação das formas \emph{\logo{{(DEUS)}TONITRUS}-hu-ta-sa} e
\emph{\logo{{(DEUS)}TONITRUS}-hu-za} talvez indique que o nome da divindade fosse
um tema consonantal em \emph{-t}, \emph{Tarhunt-}, com o nominativo
\emph{Tarhunz} \ipa{/tar.hunts/}, o mesmo valendo para a divindade solar Tiwad,
cujo nominativo seria \emph{Tiwaz} \ipa{/ti.wats/}.
\textbf{\emph{\emph{MATER}-na-tí-na}} `mãe': a leitura é garantida pelo fenício
\emph{ʾm} `mãe', pois a partir da grafia luvita, tanto \emph{anatin} `mãe'
quanto \emph{wanatin} `mulher' poderiam ser interpretados, uma vez que L.79 𔑘
FEMINA\slash{}MATER é utilizado para ambos os temas e ambos são temas em
\emph{-n-} sufixadas pelo morfema \emph{-ati-}.

\smallskip
\noindent\textbf{§IV}\tabto{2em}
\textbf{\emph{ARHA} {=?} \emph{arha-/aha-}} `completamente': é incerto se o
prevébio e advérbio representado por L.216 𔓸 ARHA era fonologicamente realizado
com a sequência \ipa{/rh/} ou com a sequência \ipa{/hh/} produzida por
assimilação, \emph{vide} hit.\ \emph{\hittitetrans{arha}} mas luv.cun.
\emph{\hittitetrans{ahha}}. Para uma discussão das formas,
ver~\citet{Yakubovich2012}.
\textbf{\emph{la+ra/i+a-nú-ha} {=?} \emph{laranuha}} `fazer prosperar?': talvez
seja uma forma causativa do verbo \emph{lada-\slash{}lara-} atestado em
AKSARAY, §2 e SULTANHAN, §6. O sentido é produzido a partir da comparação com o
hit.\ \emph{lazziya-} `prosperar', embora não esteja clara a fonologia.
A passagem em fenício contém \emph{ḥw} `fazer viver'.
Ver mais em~\citet[104--5]{HawkinsMorpurgo1978}.

\smallskip
\noindent\textbf{§V}\tabto{2em}
\textbf{\emph{\emph{“TERRA+X”}{(-)}wá/í+ra/i-za} {=?} walirin=za} `planície':
para discussão sobre a forma subjacente e troca da forma esperada
\emph{walili{(da)}-} por \emph{waliri{(da)}-},
ver~\citet[106]{HawkinsMorpurgo1978}, que também sugerem a possibilidade de
uma haplologia, i.e. *\emph{walirin=za} >
\emph{warin=za}, ou de haplografia, i.e.\ \luwiantrans{wá-ra-ra-za}
\emph{wá/í+ra/i-ra/i-za} > \luwiantrans{wá-ra-za} \emph{wá/í+ra/i-ra/i-za}.
Possível correlato de hit.\ \emph{ulili-} `campo'.
O sentido de `planície' é dado pelo fenício
\emph{ʿmq} `vale, planície'.
\textbf{\emph{zin\ldots{} zin=pa}} `de um lado\ldots{} do outro': o
ablativo-instrumental \emph{zin} tem o sentido de `aqui', a construção
contrastiva \emph{zin\ldots{} zin{(=pa)}} é comum para denotar `por um
lado\ldots{} por outro', no sentido local mas também lógico.


\smallskip
\noindent\textbf{§VI}\tabto{2em}
\textbf{\emph{tanima}} `todas': neutro plural de \emph{tanima-}, a forma neutra
plural em \emph{-aya} aparece em Ho.\ §XV\@.
\textbf{\emph{sanawiya}} `(coisas) boas = bens': neutro com sentido abstrato. A
interpretação da forma talvez seja \emph{sana-awi-} `bem-vindo', \emph{vide}
\citet{YakubovichWelcome}. Em fenício temos \emph{nʿm} `bens'.
\textbf{\emph{\emph{(“CORNU+RA/I”)}su+ra/i-sa} {=?} \emph{suras}} `abundância':
a forma subjacente não é cla\-ra, mas possivelmente esteja associada ao verbo
\emph{suwa-} `encher, preencher' (hit.\ \emph{suwai-}). A forma fenícia oferece o
sentido, \emph{šbʿ} `abundância'.
\textbf{\emph{\emph{(LINGERE)}ha-sa-sa} {=?} \emph{hasas}} `saciedade': a forma
subjacente é incerta, mas possivel\-mente seja um homônimo de \emph{hasa-} `força'
(KARKAMIŠ A11b+c, §30), que, no entanto, é acompanhada do logograma L.314 𔕡.
O logograma L.112 𔒈 LINGERE é sempre complementado por \emph{ha/há-sa/sá} e tem
o sentido de `saciedade' ou `luxo'.
O texto fenício apresenta \emph{mnʿm} `luxo'.

\smallskip
\noindent\textbf{§VIII-X}\tabto{4em}
\textbf{\emph{azun\slash{}kulanin\slash{}haralin\ldots{} azuwi\slash{}kulani\slash{}harali sara}}
`cavalo\slash{}exér\-ci\-to\slash{}escudo sobre cavalo\slash{}exército\slash{}escudo':
literalmente, as frases significam `eu fiz X sobre X', mas o sentido parece ser
de acúmulo `eu fiz X e mais X'.
Note-se que o texto fenício inverte a ordem de \emph{exército} e \emph{escudo},
Phoen. §IX \emph{mgn} `escudo' e §X \emph{mḥnt} `exército'.
O mesmo ocorre na versão hieroglífica Ho.

\smallskip
\noindent\textbf{§IX}\tabto{2em}
\textbf{\emph{\emph{EXERCITUS}-lu/a/i-za} {=?} \emph{kulanin=za}} `exército': se
aceitarmos que a for\-ma é idêntica ao luv.cun. \emph{kulana} (hit.
\emph{kuwalana-}), a melhor transliteração seria
\emph{\emph{EXERCITUS}+LU/A/I-za}, indicando que \emph{lu/a/i} age como
desambiguador fonológico e não se grafou a fonologia completa do termo.
Há também a possibilidade de se interpretar a forma subjacente como um tema em
nasal \emph{kulan-}, reforçado pela forma de ablativo
\emph{\emph{EXERCITUS}-lu/a/i-na-ti-i} {=?} \emph{kulanadi} (TELL AHMAR 6,
§24).

\smallskip
\noindent\textbf{§X}\tabto{2em}
\textbf{\emph{\emph{OMNIS}-MI-ma-za\ldots{} \emph{DEUS}-na-ri+i-ha}}: este trecho
está danificado em Hu., tendo sido reconstruído a partir da versão hieroglífica Ho.

\clearpage
\setcounter{parcount}{10}
\begin{parnumbersr}

	\raggedright%
	\itshape%


	\logo{REL}-pa-wá/í \lmasc{}\logo{(*255)}mara/i\textsuperscript{+ra/i}-ia-ní-zi
	\lmasc{}\logo{ARHA} \lmasc{}ma-ki-sa\textsuperscript{!}-há \lmasc{}\lmasc{}


	\lmasc{}\logo{(“MALUS2”)}ha-ní-ia-ta-<ia>-pa-wa/i-ta-a \lmasc{}\logo{REL}-ia
	\lmasc{}\logo{(TERRA)}ta-sà-REL+ra/i \lmasc{}a-ta \lmasc{}á-sá-ta


	\lmasc{}wá/í-ta \logo{(TERRA)}ta-sà-\logo{REL}+ra/i<-ri+i>
	\logo{ARHA} \lmasc{}\textsc{⌈}\logo{*501}\textsc{⌉} $[$…$]$-há

	\lmasc{}á-ma\lmasc{}\lmasc{}-za\textsubscript{4}-há-wá/í-ta
	\lmasc{}\logo{DOMINUS}-ní-za
	\lmasc{}\logo{DOMUS}-na-za \lmasc{}\logo{(BONUS)}sa-na-wá/í
	\lmasc{}u-sa-nú-há

	\lmasc{}á-mi-há-wa/i \lmasc{}\logo{DOMINUS}-ní-i
	\lmasc{}\logo{(NEPOS)}ha-su-a \lmasc{}\logo{OMNIS}-MI-ma
	\lmasc{}\lmasc{} \logo{(BONUS)}sa-na-wa/i-ia
	\lmasc{}\logo{CUM}-na i-zi-i-há

	\lmasc{}á-pa-sá-há-wá/í-ta \lmasc{}tá-ti-i
	\lmasc{}\logo{(“THRONUS”)}i-sà-tara/i-ti
	\lmasc{}\logo{(“SOLIUM”)}$[$i$]$-s$[$à-nu-wa/i-ha$]$

	$[$\ldots$]$

	$[$\lmasc{}\logo{OMNIS}-MI-sa-ha-wa/i-mu-ti-i \logo{REX}-ti-sa
					\lmasc{}tá-ti-na$]$ \lmasc{}$[$i-zi$]$-i-$[$da$]$
	\lmasc{}á-$[$mi$]$-ia-ti
	\lmasc{}\logo{IUSTITIA}-na-ti \lmasc{}á-mi-ia+ra/i-ha
	\lmasc{}\logo{(“COR”)}á-ta-na-sa-ma-ti \lmasc{}á-mi-ia+ra/i-há
	\lmasc{}\lmasc{} \lmasc{}\logo{(“BONUS”)}sa-na-wa/i-sa-tara/i-ti



\end{parnumbersr}

\vspace{10pt}
\hrule
\vspace{10pt}

\setcounter{parcount}{10}
\begin{parnumbersr}

	\raggedright%
	\itshape%

	kwipa=wa mariyaninzi arha makisaha,

	haniyataya=pa=wa=ta kwiya taskwiri anta asanta,

	a=wa=ta taskwirari arha parhaha.

	aman=za=ha=wa nanin=za parnan=za sanawi usanuha.

	ami=ha=wa nani \logo{NEPOS}-hasu{(w)}a tanima sanawaya \logo{CUM}-na iziha

	apasa=ha=wa=ta tati isatarati isanuwaha.

	$[$\ldots{}$]$

	tanimis=ha=wa=mu=ti hantawatis tadin izida amiyadi tarawanadi amiyari=ha
	atnasamadi amiyari=ha sanawastradi.

\end{parnumbersr}

\bigskip%
\noindent\textbf{Tradução}

[11] De fato fiz acumularem muito as colheitas dos campos-\emph{mariyana-}, [12]
enquanto os males que haviam na terra [13] eu os afastei completamente.
	[14] e a casa do meu senhor eu abençoei bem, [15] e fiz todos bens para a
descendência{(?)} do meu senhor, [16] e fi-lo sentar no trono paterno. [15]
\ldots{} [16] Todo rei me fez para si seu pai pela minha justiça e pela minha
sabedoria e pela minha bondade.

\bigskip
\noindent\textbf{Notas}

\smallskip
\noindent\textbf{§XI}\tabto{2em}
\textbf{\emph{mariyaninzi\ldots{} makisaha}} `acumulei
colheitas dos campos-\emph{mariyana}': a interpretação dessa passagem
é difícil, em parte pela presença de \emph{hapax legomena} tanto no texto luvita
quanto no texto fenício.
Sigo aqui a interpretação de \citet{VanDenHout2010}:
\textbf{\emph{mariyaninzi}}: ligada ao hitita
\textsuperscript{A.ŠÀ}\emph{mariyana-} `tipo de
campo? campo de um vegetal específico?' (KBo 10.37 12-17, 21-26), bem como às
formas luv.hier. \emph{mara/iwali-} `vegetação útil? centeio?' (SULTANHAN §6),
hit. \emph{marawalliya/i-} `campo de grãos', utilizando como evidência o uso de
L.255 𔔡 como determinativo de \emph{karunanzi} `silos (de grãos)' nesta
inscrição;
a forma escrita no texto, \emph{mariyaninzi}, deve ser interpretada como uma
forma contrata de *\emph{mariyaniyinzi}, contração da sequência \emph{-iyi-},
comum em luvita.
\textbf{\emph{makisaha}}: ligada ao hitita \emph{mekki-} `muito, numeroso' e à
passagem \emph{nu=kan ḫalkiuš EGIR-an maknunun} `eu fiz as colheitas (serem)
abundantes novamente' (Proclamação de Telipinu, KBo 3.1 iii 44, KUB 11.1 iii 8,
KBo 3.67 iii 1 + KUB 31.17:5).
Em resumo, a forma hipotética \emph{mariyaniyi-} significaria `relativo aos
campos do tipo \emph{mariyana-} > colheitas do campo-\emph{mariyana-}?' e o
verbo \emph{makisa-} seria uma forma iterativa de um verbo \emph{maki-} `fazer
crescer/abundante'.


\smallskip
\noindent\textbf{§XII}\tabto{2em}
\textbf{\emph{\emph{(“MALUS2”)}ha-ní-ia-ta-<ia>}} `males': o texto da versão
luvita Hu.\ parece ter ignorado um grafema, <\emph{ia}>, suplementado por conta
da versão Ho.\ e do pronome relativo \emph{kwiya} (nom.neut.pl.).

\smallskip
\noindent\textbf{§XIV}\tabto{2em}
\textbf{\emph{usanuwa}} `abençoar': literalmente, o verbo \emph{usanu{(wa)}-}
seria um causativo do verbo \emph{wasa-} `ser bom', logo `fazer ser
bom'. O sentido de abençoar neste contexto foi proposto pelo fato de que ao
longo do bilíngue, o fenício \emph{brk} `abençoar' é utilizado para traduzir
formas do verbo \emph{usanu{(wa)}-}.
No entanto, o texto fenício neste contexto contém o verbo \emph{yṭnʾ} `eu
ergui', o que suscitou as tentativa de interpretar \emph{usanu-} como um tema
cognato do hitita \emph{wete-} `construir', mas isso produziria um \emph{hapax
	legomena}.

\smallskip
\noindent\textbf{§XV}\tabto{2em}
\textbf{\emph{\emph{NEPOS}-hasu{(w)}a}} `descendência?': incerto, mas deve ser
um dativo singular comum.



\backmatter%

\printbibliography%

\end{document}
