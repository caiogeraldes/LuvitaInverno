
Incluo aqui todas as inscrições da fortaleza de Uradamis para aqueles que
quiserem tentar experimentar a leitura de textos por conta própria.
Os traçados e fotos das inscrições estão disponíveis no site
\href{https://www.hittitemonuments.com/hama/}{Hittite Monuments > HAMA}.
Comentários detalhados em~\citeabbrev*{CHLI11}, p.\ 411ff.



\vspace{20pt}
\noindent \textbf{HAMA 1}
\vspace{10pt}

\begin{parnumbersa}[]
	\raggedright%

	\large \luwiantrans{EGO-mi}\hspace{5pt}
	\luwiantrans{MAGNUS-ra-da-mi-sa}\hspace{5pt}
	\luwiantrans{u-ra-hi-li-na-sa}\hspace{5pt}
	\luwiantrans{FILIUS-ni-za-sa}\hspace{5pt}
	$[$\luwiantrans{i-ma-tú-wa-ni REGIO}\hspace{5pt}
					\luwiantrans{REX} $]$


	\large $[$\luwiantrans{a-wa}\hspace{5pt}
					\luwiantrans{á-mu}\hspace{5pt}
					\luwiantrans{AEDIFICARE-mi-ha}\hspace{5pt}
					\luwiantrans{za-a}$]$\hspace{5pt}
	\luwiantrans{<CASTRUM>-hara-ni-sà-za}\hspace{5pt}
	\luwiantrans{hu-ra-pa-da-wa-ni-sa REGIO}\hspace{5pt}
	\luwiantrans{FLUMEN-REGIO-da-i-sa}

	\large \luwiantrans{REL-za} \hspace{5pt}
	\luwiantrans{i-zi-i-da}\hspace{5pt}
	\luwiantrans{a-tá-ha-wa}\hspace{5pt}
	\luwiantrans{TONITRUS-HALPA-pa-wa-ni-zi REGIO}

\end{parnumbersa}

\vspace{10pt}
\hrule
\vspace{10pt}

\setcounter{parcount}{0}
\begin{parnumbersa}[]

	\raggedright%
	\itshape%

	\logo{EGO}-mi
	\logo{MAGNUS}+ra/i-da-mi-sa
	u-ra/i-hi-li-na-sa
	\logo{FILIUS}.NI-za-sa
	$[$i-ma-tú-wa/i-ni\logo{(REGIO)}
					\logo{REX}$]$

	$[$a-wa/i á-mu \logo{AEDIFICARE}+MI-ha za-'$]$ \logo{(``CASTRUM'')}hara/i-ni-sà-za
	hu+ra/i-pa-da-wa/i-ni-sa\logo{(REGIO)} \logo{FLUMEN.REGIO}-da-i-sa

	\logo{REL}-za i-zi-i-da a-tá-ha-wa/i \logo{TONITRUS.HALPA}-pa-wa/i-ni-zi\logo{(REGIO)}


\end{parnumbersa}

\vspace{10pt}
\hrule
\vspace{10pt}


\setcounter{parcount}{0}
\begin{parnumbersa}[]

	\raggedright%
	\itshape%
	amu=mi Uradamis Urhilinas nimuwizas $[$imatuwani hantawatis.$]$

	$[$a=wa amu tamaha za$]$ harnisa=za, Hurpadawanis hapadis

	kwa=za izida, anta=ha=wa Halpawaninzi.


\end{parnumbersa}

\vspace{10pt}
\hrule
\vspace{20pt}

\noindent \textbf{HAMA 3}
\vspace{10pt}

\setcounter{parcount}{0}
\begin{parnumbersa}[]
	\raggedright%
	\itshape%

	\large \luwiantrans{EGO-mi}\hspace{5pt}
	\luwiantrans{MAGNUS-ra-da-mi-sa}\hspace{5pt}
	\luwiantrans{u-ra-hi-li-na-sa}\hspace{5pt}
	\luwiantrans{FILIUS-ni-za-sa}\hspace{5pt}
	\luwiantrans{i-ma-tú-wa-ni}\hspace{5pt}
	\luwiantrans{REGIO}\hspace{5pt}
	\luwiantrans{REX}\hspace{5pt}
	\luwiantrans{a-wa}

	\luwiantrans{á-mu}\hspace{5pt}
	\luwiantrans{AEDIFICARE-mi-ha}\hspace{5pt}
	\luwiantrans{za-a}\hspace{5pt}
	\luwiantrans{<CASTRUM>-hara-ni-sà-za}\hspace{5pt}
	\luwiantrans{mu-sa-ni-pa-wa-ni-sà REGIO}\hspace{5pt}
	\luwiantrans{FLUMEN-REGIO-sà}\hspace{5pt}
	\luwiantrans{REL-za}\hspace{5pt}
	\luwiantrans{i-zi-i-da}


\end{parnumbersa}

\vspace{10pt}
\hrule
\vspace{10pt}


\setcounter{parcount}{0}
\begin{parnumbersa}[]
	\raggedright%
	\itshape%

	\logo{EGO}-mi
	\logo{MAGNUS}-ra-da-mi-sa
	u-ra-hi-li-na-sa
	\logo{FILIUS}.NI-za-sa
	i-ma-tú-wa/i-ni\logo{(REGIO)}
	\logo{REX}
	a-wa/i

	á-mu \logo{AEDIFICARE}+MI-ha za-' \logo{(``CASTRUM'')}hara/i-ni-sà-za
	mu-sa-ni-pa-wa-ni-sà\logo{(REGIO)}
	\logo{FLUMEN.REGIO}-sà
	\logo{REL}-za i-zi-i-da



\end{parnumbersa}

\vspace{10pt}
\hrule
\vspace{10pt}

\setcounter{parcount}{0}
\begin{parnumbersa}[]
	\raggedright%
	\itshape%
	amu=mi Uradamis Urhilinas nimuwizas imatuwani hantawatis.\ a=wa

	amu tamaha za harnisa=za, Musanipawanis hapadis kwa=za izida

\end{parnumbersa}

\vspace{10pt}
\hrule
\vspace{20pt}

\sloppybottom%
\clearpage

\noindent \textbf{HAMA 6}
\vspace{10pt}

\setcounter{parcount}{0}
\begin{parnumbersa}[]
	\raggedright%
	\itshape%

	\large \luwiantrans{EGO-mi}\hspace{5pt}
	\luwiantrans{MAGNUS-ra-da-mi-sa}\hspace{5pt}
	\luwiantrans{u-ra-hi-li-na-sa}\hspace{5pt}
	\luwiantrans{FILIUS-ni-za-sa}\hspace{5pt}
	\luwiantrans{i-ma-tú-wa-ni REGIO}\hspace{5pt}
	\luwiantrans{REX}\hspace{5pt}
	\luwiantrans{a-wa}\hspace{5pt}

	\large \luwiantrans{á-mu}\hspace{5pt}
	\luwiantrans{AEDIFICARE-mi-ha}\hspace{5pt}
	\luwiantrans{za-a}\hspace{5pt}
	\luwiantrans{<CASTRUM>-hara-ni-sà-za}\hspace{5pt}
	\luwiantrans{𔓻-ku-su-na-la-zi REGIO}\hspace{5pt}
	\luwiantrans{REL-za}\hspace{5pt}
	\luwiantrans{i-zi-ia-ta}


\end{parnumbersa}

\vspace{10pt}
\hrule
\vspace{10pt}


\setcounter{parcount}{0}
\begin{parnumbersa}[]
	\raggedright%
	\itshape%
	\logo{EGO}-mi
	\logo{MAGNUS}-ra-da-mi-sa
	u-ra-hi-li-na-sa
	\logo{FILIUS}.NI-za-sa
	i-ma-tú-wa/i-ni\logo{(REGIO)}
	\logo{REX}
	a-wa/i

	á-mu
	\logo{AEDIFICARE}+\emph{MI}-ha
	za-'
	\logo{(``CASTRUM'')}hara/i-ni-sà-za
	\logo{(``*218'')}ku-su-na-la-zi\logo{(REGIO)}
	\logo{REL}-za i-zi-ia-ta


\end{parnumbersa}

\vspace{10pt}
\hrule
\vspace{10pt}

\setcounter{parcount}{0}
\begin{parnumbersa}[]
	\raggedright%
	\itshape%
	amu=mi Uradamis Urhilinas nimuwizas imatuwani hantawatis.

	a=wa amu tamaha za harnisa=za, Kusunalanzi kwa=za iziyanta.


\end{parnumbersa}

\vspace{10pt}
\hrule
\vspace{20pt}


\noindent \textbf{HAMA 7}
\vspace{10pt}

\setcounter{parcount}{0}
\begin{parnumbersa}[]
	\raggedright%
	\itshape%

	\large \luwiantrans{EGO-mi}\hspace{5pt}
	\luwiantrans{MAGNUS-ra-da-mi-sa}\hspace{5pt}
	\luwiantrans{u-ra-hi-li-na-sa}\hspace{5pt}
	\luwiantrans{FILIUS-ni-za-sa}\hspace{5pt}
	\luwiantrans{i-ma-tú-wa-ni REGIO}\hspace{5pt}
	\luwiantrans{REX} \hspace{5pt}
	\luwiantrans{a-wa}\hspace{5pt}
	\luwiantrans{á-mu}\hspace{5pt}
	\luwiantrans{AEDIFICARE-mi-ha}\hspace{5pt}

	\large \luwiantrans{za-a}\hspace{5pt}
	\luwiantrans{<CASTRUM>-hara-ni-sà-za}\hspace{5pt}
	\luwiantrans{<MONS>-la-pa-ra-na-wa-ni-sa}\hspace{5pt}
	\luwiantrans{FLUMEN-REGIO-da-i-sà}\hspace{5pt}
	\luwiantrans{REL-za}\hspace{5pt}
	\luwiantrans{i-zi-i-da}\hspace{5pt}
	\luwiantrans{tú-ha-ia-ta-sa-ha REGIO}

	\large \luwiantrans{a-tá-ha-wa}\hspace{5pt}
	\luwiantrans{ha-ma-ia-ra-sa REGIO}

\end{parnumbersa}

\vspace{10pt}
\hrule
\vspace{10pt}


\setcounter{parcount}{0}
\begin{parnumbersa}[]
	\raggedright%
	\itshape%
	\logo{EGO}-mi
	\logo{MAGNUS}-ra-da-mi-sa
	u-ra-hi-li-na-sa
	\logo{FILIUS}.NI-za-sa
	i-ma-tú-wa/i-ni\logo{(REGIO)}
	\logo{REX}
	a-wa/i
	á-mu
	\logo{AEDIFICARE}+MI-ha

	za-'
	\logo{``CASTRUM''}hara/i-ni-sà-za
	\logo{``MONS''}.la-pa+ra/i-na-wa/i-ni-sa
	\logo{FLUMEN.REGIO}-da-i-sà
	\logo{REL}-za
	i-zi-i-da
	tú-ha-ia-ta-sa-ha\logo{(REGIO)}

	a-tá-ha-wa/i ha-ma-ia+ra/i-sa\logo{(REGIO)}



\end{parnumbersa}

\vspace{10pt}
\hrule
\vspace{10pt}


\setcounter{parcount}{0}
\begin{parnumbersa}[]
	\raggedright%
	\itshape%

	amu=mi Uradamis Urhilinas nimuwizas imatuwani hantawatis. a=wa amu tamaha

	za harnisa=za, Labarnawanis hapadis kwa=za izita,
	Tuhayatas=ha

	anta=ha=wa Hamayaras.


\end{parnumbersa}

\vspace{10pt}
\hrule
\vspace{20pt}



\begin{multicols}{2}[\noindent\textbf{Vocabulário}]
	\begin{hangparas}{1em}{1}
		\raggedright%
		\textbf{\emph{hamayarawani}-} (\emph{adj.}) \tabto{1em} proveniente de Hamayara\\
		\textbf{\emph{hurpadawani}-} (\emph{adj.}) \tabto{1em} proveniente de Hurpada\\
		\textbf{\emph{kusunala}-} (\emph{adj.}) \tabto{1em} proveniente de Kusuna\\
		\textbf{\emph{labarnawani}-} (\emph{adj.}) \tabto{1em} proveniente de Labarna\\
		\textbf{\emph{musanipawani}-} (\emph{adj.}) \tabto{1em} proveniente de Musanipa\\
		\textbf{\emph{tuhayata}-} (TO) \tabto{1em} Tuhayata\\
	\end{hangparas}
\end{multicols}

\vfill
