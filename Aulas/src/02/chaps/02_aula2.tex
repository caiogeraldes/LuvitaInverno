% TeX root=../main.tex

\chapter{Sistema verbal}

\section{Flexão}

Até onde temos atestação no \emph{corpus}, as formas finitas do verbo luvita 
flexionam em:
\begin{inparaenum}[(a)]
	\item voz: ativa e médio-passiva;
	\item tempo: presente e pretérito;
	\item modo: indicativo e imperativo.
\end{inparaenum}
Além das formas finitas, também temos em luvita o infinitivo, o gerundivo, uma 
forma de substantivo verbal e particípios na voz ativa e passiva.

\paragraph{Desinências do indicativo}
A tabela a seguir contém as desinências do indicativo.\footnote{Nas tabelas a
seguir, as formas em colchetes são particularmente raras. As formas com {?} não
são atestadas.}
As formas médias terminadas em \emph{-si} talvez representem a adição de um
pronome reflexivo \emph{-si}, não atestada em nenhum outro contexto.

\begin{center}
	\begin{tabular}[c]{lll|ll}
		\toprule
		&\multicolumn{2}{c|}{Presente do indicativo}&\multicolumn{2}{c}{Pretérito do indicativo}\\
		& ativo & médio-passivo & ativo & médio-passivo\\
		\midrule
		1sg. & \emph{-wi}                                            & {?}                            & \emph{-ha}       & \emph{-hasi}              \\
		2sg. & \emph{-si} [\emph{-tis}]                              & \emph{-ta}                     & {?}              &                           \\
		3sg. & \emph{-ti}\slash{}\emph{-ri}, [\emph{-i}, \emph{-ia}] & \emph{-ati}\slash{}\emph{-ari} & \emph{-ta}       & \emph{-asi}, \emph{-tasi} \\
		1pl. & {?}                                                   & {?}                            & \emph{-han}{(?)} & {?}                       \\
		2pl. & \emph{-tani}                                          & {?}                            & \emph{-tan}      & {?}                       \\
		3pl. & \emph{-nti}                                           & {?}                            & \emph{-nta}      & \emph{-antasi}            \\
		\bottomrule
	\end{tabular}
\end{center}

\noindent As formas de 3sg.ind. \emph{-ri} e 2pl.ind. \emph{-rani} são
rotacizadas.

\paragraph{Desinências do imperativo}
A tabela a seguir contém as desinências do imperativo.

\begin{center}
	\begin{tabular}[c]{lll}
		\toprule
		&\multicolumn{2}{c}{Imperativo}\\
		& ativo & mp.\\
		\midrule
		2sg. &  \emph{∅}                           & {?} \\
		3sg. & \emph{-tu} & \emph{-aru} \\
		\midrule
		2pl. & \emph{-ranu}                                          & {?}\\
		3pl. & \emph{-ntu}                                           & {?} \\
		\bottomrule
	\end{tabular}
\end{center}

\noindent A forma 2pl.imp. \emph{-ranu} é rotacizada de uma forma não atestada
*\emph{-tanu}.

\paragraph{Formas não-finitas}

\begin{compactitem}
\item Particípio
	\begin{compactitem}
	\item Ativo: \emph{-ant{(i)}-}
	\item Passivo: \emph{-ama\slash{}i-}
	\end{compactitem}
\item Substantivo verbal: \emph{-ur-}	
\item Infinitivo: \emph{-una}
\item Gerundivo: \emph{-min{(a)}}
\end{compactitem}

\section{Morfologia derivacional}

\paragraph{Sufixos}
Algumas formas verbais são produzidas por derivação, utilizando os seguintes
sufixos:
\begin{compactenum}[(a)]
\item \emph{-sa-}: sentido iterativo:
	\begin{center}
	\emph{maranuha} `eu destrui' (KARKAMIŠ A1a, §9)\\
		$\downarrow$\\
	\emph{maranu\textbf{sa}ha} `eu destruí várias vezes' (TELL AHMAR 6, §6)
	\end{center}
\item \emph{-za-}: sentido iterativo:
	\begin{center}
	\emph{waliyanta} `eles ergueram' (KARKAMIŠ A14a, §§6, \textsc{⌈}7\textsc{⌉})\\
		$\downarrow$\\
	\emph{waliya\textbf{za}nta} `eles ergueram (repetidamente)' (IZGIN 1, §18)
	\end{center}
\item \emph{-nu{(wa)}-}: sentido causativo:
	\begin{center}
	\emph{taha} `eu ergui' (ARSUZ 1+2, §§9)\\
		$\downarrow$\\
	\emph{ta\textbf{nu}ha} `eu fiz erguer' (KARKAMIŠ A6, §19)
	\end{center}
\end{compactenum}

\paragraph{Redobro}
O redobro é utilizado por vezes para produzir o sentido iterativo:
\begin{center}
	\emph{sarlati} `ele oferece' (ANCOZ 9, §2)\\
	$\downarrow$\\
	\emph{\textbf{sa}sarlai} `ele sempre oferece' (BULGARMADEN, §11)
\end{center}

\paragraph{Prevérbios}
Prevérbios são preposições que alteram o sentido do verbo.
As mais comuns são:
\begin{compactenum}[(a)]
\item *\emph{anan} `abaixo, para baixo'
\item \emph{anta} `em, dentro'
\item \emph{antan} `para dentro'
\item \emph{apan{(i)}} `atrás (de)'
\item \emph{arha} `completamente, embora'
\item CUM\emph{-ni/-i} `?'
\item *\emph{kata} `para baixo'
\item \emph{paran{(i)}} `na frente de'
\item \emph{pari} `por cima'
\item \emph{sara} `para cima'
\end{compactenum}
	 
\chapter{Clíticos}

\chapter{Leitura: HAMA 2}
