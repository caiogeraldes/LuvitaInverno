% TeX root=../main.tex

\chapter{Sistema verbal}

\section{Flexão}

Até onde temos atestação no \emph{corpus}, as formas finitas do verbo luvita
flexionam em:
\begin{inparaenum}[(a)]
	\item voz: ativa e médio-passiva;
	\item tempo: presente e pretérito;
	\item modo: indicativo e imperativo.
\end{inparaenum}
Além das formas finitas, também temos em luvita o infinitivo, o gerundivo, uma
forma de substantivo verbal e particípios na voz ativa e passiva.

\paragraph{Desinências do indicativo}
A tabela a seguir contém as desinências do indicativo.\footnote{Nas tabelas a
	seguir, as formas em colchetes são particularmente raras. As formas com {?} não
	são atestadas.}
As formas médias terminadas em \emph{-si} talvez representem a adição de um
pronome reflexivo \emph{-si}, não atestada em nenhum outro contexto.

\begin{center}
	\begin{tabular}[c]{lll|ll}
		\toprule
		     & \multicolumn{2}{c|}{Presente do indicativo}           & \multicolumn{2}{c}{Pretérito do indicativo}                                                \\
		     & ativo                                                 & médio-passivo                               & ativo            & médio-passivo             \\
		\midrule
		1sg. & \emph{-wi}                                            & {?}                                         & \emph{-ha}       & \emph{-hasi}              \\
		2sg. & \emph{-si} [\emph{-tis}]                              & \emph{-ta}                                  & {?}              &                           \\
		3sg. & \emph{-ti}\slash{}\emph{-ri}, [\emph{-i}, \emph{-ia}] & \emph{-ati}\slash{}\emph{-ari}              & \emph{-ta}       & \emph{-asi}, \emph{-tasi} \\
		1pl. & {?}                                                   & {?}                                         & \emph{-han}{(?)} & {?}                       \\
		2pl. & \emph{-tani}                                          & {?}                                         & {?}              & {?}                       \\
		3pl. & \emph{-nti}                                           & {?}                                         & \emph{-nta}      & \emph{-antasi}            \\
		\bottomrule
	\end{tabular}
\end{center}

\noindent As formas de 3sg.pres.atv. \emph{-ri} e 2pl.pres.atv. \emph{-rani} são
rotacizadas. A forma 1pl.pres.atv. \emph{-han} talvez seja uma forma singular,
conforme proposto por~\citet{Carruba1984} \emph{contra}~\citet{MorpurgoDavies1980}.
Autores mais antigos interpretaram incorretamente a desinência
gerundiva \emph{-min{(a)}} como 1pl.pres.atv.

\paragraph{Desinências do imperativo}
A tabela a seguir contém as desinências do imperativo.

\begin{center}
	\begin{tabular}[c]{lll}
		\toprule
		     & \multicolumn{2}{c}{Imperativo}               \\
		     & ativo                          & mp.         \\
		\midrule
		2sg. & \emph{∅}                       & {?}         \\
		3sg. & \emph{-tu}                     & \emph{-aru} \\
		\midrule
		2pl. & \emph{-ranu}                   & {?}         \\
		3pl. & \emph{-ntu}                    & {?}         \\
		\bottomrule
	\end{tabular}
\end{center}

\noindent A forma 2pl.imp. \emph{-ranu} é rotacizada de uma forma não atestada
*\emph{-tanu}.

\paragraph{Formas não-finitas}
As formas não finitas atestada são:

\begin{compactitem}
	\item Particípio passivo: \emph{-ama\slash{}i-}\footnote{Talvez haja uma única
		atestação de um particípio passivo em \emph{-ant-}: \emph{harwatanza}
		`viajando' (JISR EL HADID 4, §4).}
	\item Substantivo verbal: \emph{-ur-}
	\item Infinitivo: \emph{-una}
	\item Gerundivo: \emph{-min{(a)}}
\end{compactitem}



\section{Quadro de conjugação}

O quadro a seguir contem a conjugação do verbo \emph{izi(ya)-} `fazer', com as
formas do verbo \emph{la-} `pegar', \emph{pi-} `dar', \emph{as-} `ser', \emph{hwihwisa-}
`correr' e \emph{tumanti-} `escutar' onde necessário por falta de atestação.

\begin{center}
	\begin{tabular}[c]{lll|ll}
		\toprule
		     & \multicolumn{2}{c|}{Pres.\ ind.} & \multicolumn{2}{c}{Pret.\ ind.}                                            \\

		     & atv.\emph{}                      & mp\emph{}
		     & atv.\emph{}                      & mp\emph{}                                                                  \\
		\midrule
		1sg. & \emph{iziyawi}                   & {?}\emph{}                      & \emph{iziyaha}     & \emph{izihasi}      \\
		2sg. & \emph{lasi}                      & \emph{piyata}                   & {?}\emph{}         & {?}                 \\
		3sg. & \emph{iziti, piyai}              & \emph{iziyari}                  & \emph{izita}       & \emph{hwihwisatasi} \\
		1pl. & {?}                              & {?}                             & \emph{izihan}{(?)} & {?}                 \\
		2pl. & \emph{asatani}                   & {?}                             & {?}                & {?}                 \\
		3pl. & \emph{iziyanta}                  & {?}                             & \emph{piyanta}     & \emph{iziyantasi}   \\
		\bottomrule
	\end{tabular}
\end{center}


\begin{center}
	\begin{tabular}[c]{lll}
		\toprule
		                                        & \multicolumn{2}{c}{Imp.}                  \\
		\midrule
		2sg.                                    & \emph{iziya}             & {?}            \\
		3sg.                                    & \emph{iziyatu}           & \emph{iziyaru} \\
		2pl.                                    & \emph{tumantiranu}       & {?}            \\
		3pl.                                    & \emph{iziyantu}          & {?}            \\
		\midrule
		\midrule
		\multicolumn{2}{l}{Particípio  passivo} & {\emph{tumantimi-}}                       \\
		\multicolumn{2}{l}{Infinitivo}          & {\emph{lana}}                             \\
		\multicolumn{2}{l}{Gerundivo}           & {\emph{iziyamin{(a)}}}                    \\
		\bottomrule
	\end{tabular}
\end{center}

\section{Morfologia derivacional}

\paragraph{Sufixos}
Algumas formas verbais são produzidas por derivação, utilizando os seguintes
sufixos:
\begin{compactenum}[(a)]
	\item \emph{-sa-}: sentido iterativo:
	\begin{center}
		\emph{maranuha} `eu destrui' (KARKAMIŠ A1a, §9)\\
		$\downarrow$\\
		\emph{maranu\textbf{sa}ha} `eu destruí várias vezes' (TELL AHMAR 6, §6)
	\end{center}
	\item \emph{-za-}: sentido iterativo:
	\begin{center}
		\emph{waliyanta} `eles ergueram' (KARKAMIŠ A14a, §§6, \textsc{⌈}7\textsc{⌉})\\
		$\downarrow$\\
		\emph{waliya\textbf{za}nta} `eles ergueram (repetidamente)' (IZGIN 1, §18)
	\end{center}
	\item \emph{-nu{(wa)}-}: sentido causativo:
	\begin{center}
		\emph{taha} `eu ergui' (ARSUZ 1+2, §§9)\\
		$\downarrow$\\
		\emph{ta\textbf{nu}ha} `eu fiz erguer' (KARKAMIŠ A6, §19)
	\end{center}
\end{compactenum}

\paragraph{Redobro}
O redobro é utilizado por vezes para produzir o sentido iterativo:
\begin{center}
	\emph{sarlati} `ele oferece' (ANCOZ 9, §2)\\
	$\downarrow$\\
	\emph{\textbf{sa}sarlai} `ele sempre oferece' (BULGARMADEN, §11)
\end{center}

\paragraph{Prevérbios}
Prevérbios são preposições que alteram o sentido do verbo.
As mais comuns são:
\begin{compactenum}[(a)]
	\item *\emph{anan} `abaixo, para baixo'
	\item \emph{anta} `em, dentro'
	\item \emph{antan} `para dentro'
	\item \emph{apan{(i)}} `atrás (de)'
	\item \emph{arha} `completamente, embora'
	\item CUM\emph{-ni/-i} `?'
	\item *\emph{kata} `para baixo'
	\item \emph{paran{(i)}} `na frente de'
	\item \emph{pari} `por cima'
	\item \emph{sara} `para cima'
\end{compactenum}

\chapter{Clíticos}

\chapter{Leitura: HAMA 2}
