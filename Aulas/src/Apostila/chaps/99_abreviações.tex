% TeX root=../main.tex
\newcommand{\abreviacao}[2]{\noindent #1 \hspace{10pt}\hfill #2\\}
\newenvironment{abreviacoes}{
	\flushright%
	\setlength{\columnsep}{50pt}
	\small
	\begin{multicols}{2}
		}{
	\end{multicols}
}

\section*{Geral}
\begin{abreviacoes}
	\abreviacao{\textsc{aec}}{antes da era comum}
	\abreviacao{C}{consoante}
	\abreviacao{\emph{ca.}}{\emph{circa}}
	\abreviacao{CHLI}{\emph{\foreignlanguage{english}{Corpus of Hieroglyphic Luwian Inscriptions}}}
	\abreviacao{CTH}{\emph{\foreignlanguage{german}{Catalog der Texte der Hethiter}}}
	\abreviacao{GrHL}{\emph{\foreignlanguage{english}{A Grammar of the Hittite Language}}}
	\abreviacao{hit.}{hitita}
	\abreviacao{KBo}{\emph{\foreignlanguage{german}{Keilschrifttexte aus Boghazköi}}}
	\abreviacao{KUB}{\emph{\foreignlanguage{german}{Keilschrifturkunden aus Boghazköi}}}
	\abreviacao{\La{$x$}}{Sinal número $x$}
	\abreviacao{\textsc{pie}}{Proto-indo-europeu}
	\abreviacao{RlA}{\emph{\foreignlanguage{german}{Reallexikon der Assyriologie}}}
	\abreviacao{StuBoT}{\emph{\foreignlanguage{german}{Studien zu den
				Bogazköy-Texten}}}
	\abreviacao{V}{vogal}
\end{abreviacoes}


\section*{Glosas e vocabulário}
\begin{abreviacoes}
	\abreviacao{abl.}{ablativo}
	\abreviacao{acu.}{acusativo}
	\abreviacao{adj.}{adjetivo}
	\abreviacao{clt.}{clítico}
	\abreviacao{com.}{gênero comum}
	\abreviacao{conj.}{conjunção}
	\abreviacao{dat.}{dativo}
	\abreviacao{det.}{determinativo}
	\abreviacao{gen.}{genitivo}
	\abreviacao{neut.}{gênero neutro}
	\abreviacao{nom.}{nominativo}
	\abreviacao{NP}{nome próprio}
	\abreviacao{pl.}{plural}
	\abreviacao{prep.}{preposição}
	\abreviacao{pret.}{pretérito}
	\abreviacao{refl.}{pronome reflexivo}
	\abreviacao{rel.}{pronome relativo}
	\abreviacao{sg.}{singular}
	\abreviacao{subst.}{substantivo}
	\abreviacao{TO}{topônimo}
	\abreviacao{v.}{verbo}
	\abreviacao{v.t.}{v.\ transitivo}
	\abreviacao{$X$.}{NP iniciado por $X$}
	\abreviacao{pro.}{pronome}
\end{abreviacoes}
