% TeX root=../main.tex
\newcommand{\abreviacao}[2]{\noindent #1 \hspace{10pt}\hfill #2\\}
\newenvironment{abreviacoes}{
	\flushright%
	\setlength{\columnsep}{50pt}
	\small
	\begin{multicols}{2}
		}{
	\end{multicols}
}

\section*{Geral}
\begin{abreviacoes}
	\abreviacao{\textsc{aec}}{antes da era comum}
	\abreviacao{C}{consoante}
	\abreviacao{\emph{ca.}}{\emph{circa}}
	\abreviacao{CTH}{\emph{\foreignlanguage{german}{Catalog der Texte der Hethiter}}}
	\abreviacao{hit.}{hitita}
	\abreviacao{KBo}{\emph{\foreignlanguage{german}{Keilschrifttexte aus Boghazköi}}}
	\abreviacao{\La{$x$}}{Sinal número $x$}
	\abreviacao{\textsc{pie}}{Proto-indo-europeu}
	\abreviacao{StuBoT}{\emph{\foreignlanguage{german}{Studien zu den
				Bogazköy-Texten}}}
	\abreviacao{V}{vogal}
\end{abreviacoes}


\section*{Glosas e vocabulário}
\begin{abreviacoes}
	\abreviacao{abl.}{Ablativo}
	\abreviacao{acu.}{Acusativo}
	\abreviacao{adj.}{adjetivo}
	\abreviacao{clt.}{Clítico}
	\abreviacao{com.}{Gênero comum}
	\abreviacao{dat.}{Dativo}
	\abreviacao{det.}{Determinativo}
	\abreviacao{gen.}{Genitivo}
	\abreviacao{neut.}{Gênero neutro}
	\abreviacao{nom.}{Nominativo}
	\abreviacao{NP}{Nome próprio}
	\abreviacao{pl.}{Plural}
	\abreviacao{prep.}{Preposição}
	\abreviacao{sg.}{Singular}
	\abreviacao{subst.}{substantivo}
	\abreviacao{v.}{verbo}
	\abreviacao{v.t.}{v.\ transitivo}
	\abreviacao{$X$.}{NP iniciado por $X$}
\end{abreviacoes}
