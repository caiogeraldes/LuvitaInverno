% TeX root=../main.tex

\section{Proto-Anatólico Comum}

Esta seção se dedica a apresentar o panorama geral da fonologia do
proto-ana\-tó\-lico comum, mostrando os desenvolvimentos linguísticos
observáveis a partir da nossa reconstrução do proto-indoeuropeu.
As seções dedicadas à fonologia do proto-anatólico comum e luvita
são baseadas em~\citet[53--91]{Melchert1994}, com algumas adições
de~\citet{HSK41.2}.

\subsection{Consoantes}

\begin{flushleft}
	\begin{tabular}[c]{lllllll}
		\multirow[t]{2}{*}{\textbf{Oclusivas}}  & \textbf{Surdas}  & \ipa{*/p/} & \ipa{*/t/} & \ipa{*/k/}  & */\pietrans{c}/ & \ipa{*/\pietrans{kw}/} \\
		                                        & \textbf{Sonoras} & \ipa{*/b/} & \ipa{*/d/} & \ipa{*/g/}  & */\pietrans{j}/ & \ipa{*/\pietrans{gw}/} \\[2ex]
		\textbf{Africadas}                      & \textbf{Surdas}  &            &            & \ipa{*[ts]} &                 &                        \\[2ex]
		\multirow[t]{2}{*}{\textbf{Fricativas}} & \textbf{Surdas}  &            & \ipa{*/s/} &             & \ipa{*/H/}      &                        \\
		                                        & \textbf{Sonoras} &            &            &             & \ipa{*/h/}      &                        \\[2ex]

		\textbf{Sonorantes}                     &                  & \ipa{*/r/} & \ipa{*/l/} & \ipa{*/w/}  & \ipa{*/y/}      &                        \\
		\textbf{Nasais}                         &                  & \ipa{*/m/} & \ipa{*/n/} &             &                 &                        \\
	\end{tabular}
\end{flushleft}



\begin{compactitem}
	\item As oclusivas sonoras aspiradas \pie{}
	\ipa{*/\pietrans{bh, dh, jh, gh, gwh}/} colapsaram nas oclusivas sonoras
	\pac~\ipa{*/\pietrans{b, d, j, g, gw}/}.
	\item As línguas anatólicas não preservam reflexos diretos da laringal
	\ipa{*/\pietrans{h1}/} ou da laringal \ipa{*/\pietrans{h3}/} não-inicial, mas
	evidências indiretas permitem assumir um desenvolvimento distinto para
	\ipa{*/\pietrans{h2}/}, em uma fricativa faríngea ou dorsal surda \ipa{*/H/}
	e uma ou mais sonoras \ipa{*/h/} para as demais laringais (em alguns autores
	separada em três fricativas distintas: \ipa{*[ħ],*/hʷ/ */\pietrans{h3}/}).
	\item a africada \ipa{*/ts/} talvez ainda fosse um alofone de \ipa{*/t/} antes
	de \ipa{*/y/}.
	\item Há evidência que as sonorantes \ipa{*/r, l/} e as nasais \ipa{*/m, n/}
	ainda ocorriam em núcleo silábico: \ipa{*/\pietrans{R, L, M, N}/}
\end{compactitem}


\subsection{Vogais}


\begin{flushleft}
	\begin{tabular}[c]{lllllll}
		\ipa{*/i/} & \ipa{*/i:/} &              &            &            & \ipa{*/u/} & \ipa{*/u:/} \\
		           &             & \ipa{*/ẹ:/}? &            &            &            &             \\
		           & \ipa{*/e/}  & \ipa{*/e:/}  &            & \ipa{*/o/} & \ipa{*/o/} &             \\
		           &             & \ipa{*/æ:/}  &            &            &            &             \\
		           &             & \ipa{*/a/}   & \ipa{*/a/} &            &            &             \\
	\end{tabular}
\end{flushleft}

\begin{compactitem}
	\item \ipa{*/ẹ:/} representa o resultado de monotongação do
	\pie~\ipa{*/\pietrans{ey}/}.
	\item \ipa{*/æ/} representa o resultado de \pie~\ipa{*/\pietrans{eh1}/}
	(tautossilábico).
\end{compactitem}

\subsection{Do proto-indoeuropeu ao proto-anatólico}

Notar que nesta seção, os exemplos do hitita, palaico e luvita cuneiforme
utilizarão a série <\ipa{\emph{p, t, k}}> para representar as oclusivas
\emph{lenes}\slash{}sonoras e a <\ipa{\emph{pp, tt, kk}}> para representar as
oclusivas \emph{fortes}\slash{}surdas, por conta das idiossincrasias do uso do
cuneiforme pelos escribas de Hattusa.

\subsubsection{Vogais}

As vogais no geral parecem ter sido preservadas em~\pac.
As principais mudanças são:
\begin{compactitem}
	\item \pie~\ipa{*/\emph{\pietrans{ey}}/} > \pac~\ipa{*/\emph{e}̣̄/}
	\item  \pie~\ipa{*/\emph{\pietrans{eh1}}/} > \pac~\ipa{*/\emph{ǣ}/} /
	\item  \pie~\ipa{*/\emph{\pietrans{Vh}\textsubscript{1,3}}/} >
	\pac~\ipa{*/\emph{V̄}/} /
	\ipa{\_σ}
	\item \pie~\ipa{*/\emph{\pietrans{ew}}/} > \pac~\ipa{*/\emph{ū}/} ou algo
	próximo de \ipa{*/\emph{ọ̄}/}
	\item vogais longas originalmente não acentuadas são abreviadas
\end{compactitem}

\subsubsection{Oclusivas}

As principais mudanças que ocorreram com as oclusivas do \pie~são:

\begin{flushleft}
	\textbullet~\pie~\ipa{*/\emph{\pietrans{bh, dh, jh, gh, gwh}}/} >
	\pac~\ipa{*/\emph{b, d, \pietrans{j, g, gw}}/}\footnote{Falta de evidência positiva para existência
		de uma série de aspiradas sonoras em \pac.}

	\textbullet~\pie~\ipa{*C $[\text{surda}]$} > \pac~\ipa{*C $[\text{sonora}]$}
	/ \ipa{V̄́}\_
	\begin{compactitem}
		\item hit.\ \emph{iēzzi} < \pac~\ipa{\emph{*Hǣ-di}} <
		\pie~\ipa{\emph{*\pietrans{Hyéh1-ti}}}
		\item luv.cun.\ \emph{āta}, luv.hier.\
		\emph{ada, ara}, líc.\ \emph{ade}
		`ele fez' < \pac~\ipa{\emph{*Hǣ-do}} <
		\pie~\ipa{\emph{*\pietrans{Hyéh1-to}}}
	\end{compactitem}

	\textbullet~\pie~\ipa{*C $[\text{surda}]$} > \pac~\ipa{*C $[\text{sonora}]$} /
	\ipa{—́V}\_\ipa{V}
	\begin{compactitem}
		\item luv.cun.\ \emph{-ati};
		luv.hier.\ \emph{-adi,-ari},
		líc.\ \emph{-e,-adi},
		líd.\ \emph{-ad}? desinência de ablativo\slash{}instrumental
		< \pac~\ipa{*\emph{—́odi}} < \pie~\ipa{*\emph{—́oti}}
	\end{compactitem}

	\textbullet~\pie~\ipa{*\emph{\pietrans{kw}}} >
	\pac~\ipa{*\emph{\pietrans{gw}}} em posição medial\footnote{Fonologicamente sem motivação, mas é
		a única descrição possível}
	\begin{compactitem}
		\item hit.\ \emph{tarku-};
		luv.cun.\ \emph{taru-} `dança'
		< \pac~\ipa{*\emph{\pietrans{tergw-}}}
		< \pie~\ipa{*\emph{\pietrans{terkw-}}} `torcer'
	\end{compactitem}

	\textbullet~\pie~\ipa{*\emph{\pietrans{kw}}} é retido antes de \ipa{*\emph{s}} e do morfema iterativo
	\ipa{*\emph{\pietrans{-sce/o-}}}:
	\begin{compactitem}
		\item hit.\ \emph{\hittitetrans{tekkussa-}} `mostrar' <
		\pie~\ipa{*\emph{\pietrans{dekwso-}}}.
	\end{compactitem}

\end{flushleft}

\subsubsection{Laringais}

A laringal \pie~\ipa{*\emph{\pietrans{h1}}} desaparece nas línguas anatólicas.
A laringal \pie~\ipa{*\emph{\pietrans{h2}}} resulta em dois alofones distintos,
um surdo: \ipa{\emph{h}} e um sonoro \ipa{\emph{h}/\emph{ħ}}
\begin{compactitem}
	\item \pie~\ipa{*\emph{\pietrans{h2}}} > \pac~\ipa{*\emph{H}} / \ipa{V̆́\_}
	\item \pie~\ipa{*\emph{\pietrans{h2}}} > \pac~\ipa{*\emph{h}/\emph{ħ}} / \ipa{V̄́\_V}
\end{compactitem}
A distinção entre \pie~\ipa{*\emph{\pietrans{h2}}} e \ipa{*\emph{\pietrans{h3}}}
aparece preservada em lício e \pie~\ipa{*\emph{\pietrans{h3}}} parece estar
preservada em posições iniciais em luvita, de modo que se supõe o desenvolvimento:
\begin{compactitem}
	\item \pie~\ipa{*\emph{\pietrans{h2}}} > \pac~\ipa{*\emph{h}} / \ipa{\#\_}
	\begin{compactitem}
		\item \pac~\ipa{*\emph{h}} > líc. \emph{x, q, k} / \#\_
	\end{compactitem}
	\item \pie~\ipa{*\emph{\pietrans{h3}}} > \pac~\ipa{*\emph{\pietrans{h3}}}
	\begin{compactitem}
		\item \pac~\ipa{*\emph{\pietrans{h3}}} > líc. \ipa{∅} / \#\_
		\item \pac~\ipa{*\emph{\pietrans{h3}}} > luv. \emph{h} / \#\_
	\end{compactitem}
\end{compactitem}
Supõe-se que uma laringal labializada \ipa{*/\emph{hʷ}/} surge em~\pac~a partir
da sequência \pie~\ipa{*\emph{\pietrans{h2w}}} e \ipa{*\emph{\pietrans{h3w}}}.

\subsubsection{Nasais}

As nasais \emph{m} e \emph{n} convergem em posição final em todas as línguas.
A sequência \ipa{*\emph{NH}} produz geminação da nasal.
Em posição de núcleo silábico, o desenvolvimento é: \ipa{*\emph{N̥}} >
*\ipa{\emph{aN}}.

\subsubsection{Resoantes}
Em posição de núcleo silábico, o desenvolvimento é: \ipa{*\emph{R̥}} >
*\ipa{\emph{aR}}.
As líquidas \ipa{*\emph{r}} e \ipa{*\emph{l}} são preservadas e aparecem
geminadas como resultado de \emph{LN} ou \emph{LH}.
As semivogais são preservadas, mas o \ipa{*/\emph{\pietrans{y}}/} inicial cai
antes de \pac~\ipa{*/\emph{\pietrans{e}}/}, \ipa{*/\emph{ē}/}, e
\ipa{*/\emph{ǣ}/}.

\subsubsection{Sibilante \ipa{*\emph{s}}}

A sibilante \pie~\ipa{*\emph{s}} é preservada na grande maioria de contextos em
\pac.
A sequência \ipa{*\emph{sT}} é preservada em hitita, possivelmente com a
inclusão de uma vogal protética /\emph{i}/ que talvez seja apenas uma
representação gráfica para o encontro consonantal: /\emph{sTV}/ =
<\emph{\hittitetrans{is-TV}}>.

A geminação \emph{-ss-} do hitita, como as demais geminações e sonorizações da
língua,
parece ser resultado de uma espécie de lei de Čop, em que a sequẽncia
\pie~\ipa{*\emph{ĕ́.C\textsubscript{1}V}} >
\pac~\ipa{*\emph{aC\textsubscript{1}.C\textsubscript{1}V}}, regra que parece ter
agido \textbf{após} a queda da laringal \ipa{*\emph{\pietrans{h1}}}:
\pie~\ipa{*\emph{\pietrans{h1ésu-}}} > \ipa{*\emph{é.su-}} >
\pac~\ipa{*\emph{ás.su}} > hit. \emph{\hittitetrans{assu-}}
`bom'.\footnote{O contraste entre <\emph{\hittitetrans{Vs-sV}}> e
	<\emph{\hittitetrans{V-sV}}> não tem interpretação fonológica clara como no
	caso das consoantes oclusivas em que se presume que a forma duplicada
	representa uma consoante surda e a forma simples uma consoante sonora.}


\section{Luvita}

\subsection{Consoantes}

\begin{flushleft}
	\begin{tabular}[c]{llllll}
		\multirow[t]{2}{*}{\textbf{Oclusivas}}  & \textbf{Surdas}  & \ipa{*/p/} & \ipa{*/t/}  & \ipa{*/k/} &            \\
		                                        & \textbf{Sonoras} & \ipa{*/b/} & \ipa{*/d/}  & \ipa{*/g/} &            \\[2ex]
		\textbf{Africadas}                      & \textbf{Surdas}  &            & \ipa{*[ts]} &                         \\[2ex]
		\multirow[t]{2}{*}{\textbf{Fricativas}} & \textbf{Surdas}  & \ipa{*/s/} &             & \ipa{*/H/} &            \\
		                                        & \textbf{Sonoras} &            &             & \ipa{*/h/} &            \\[2ex]
		\textbf{Sonorantes}                     &                  & \ipa{*/r/} & \ipa{*/l/}  & \ipa{*/w/} & \ipa{*/y/} \\
		\textbf{Nasais}                         &                  & \ipa{*/m/} & \ipa{*/n/}  &            &            \\
	\end{tabular}
\end{flushleft}

\begin{flushleft}
	\textbullet~Oclusivas surdas foram generalizadas para a posição inicial
	\textbullet~\ipa{*\emph{n}} inicial muda para uma consoante nasal
	grafada com <\emph{t}> tanto em cuneiforme quanto hieroglífico de maneira
	irregular.
\end{flushleft}


\subsection{Vogais}

\begin{flushleft}
	\begin{tabular}[c]{lllllll}
		\ipa{*/i/} & \ipa{*/i:/} &            &            &  & \ipa{*/u/} & \ipa{*/u:/} \\
		           &             & \ipa{*/a/} & \ipa{*/a/} &  &            &             \\
	\end{tabular}
\end{flushleft}

\subsection{Do proto-anatólico ao luvita}

Nos exemplos utilizados nesta seção, a vogal longa representa a evidência
produzida a partir do luvita cuneiforme.

\subsubsection{Oclusivas}

As oclusivas são, em sua maioria, preservadas.
Oclusivas surdas foram generalizadas para a posição inicial, embora isto dependa
da nossa interpretação do cuneiforme.
As lábio-velares
\ipa{*\emph{\pietrans{gw}}} e \ipa{*\emph{\pietrans{kw}}} se convertem,
respectivamente, em \ipa{\emph{w}} e \ipa{\emph{k\pietrans{w}}}.

A palatal \pac~\ipa{*/\emph{\pietrans{c}}/} se desenvolve na africada \emph{ts}
do luvita de maneira incondicional, colidindo com o \emph{ts} produzido pelo
encontro de \emph{t+s}:
\pac~\ipa{*\emph{\pietrans{co-}}/\emph{\pietrans{c{(o)}i-}}} `este' >
\emph{za-}, \emph{zi-};
\pac~\ipa{*\emph{\pietrans{-isce/o-}}} \emph{iterativo} >
\emph{-z{(z)}a-};
\pac~\ipa{*\emph{\pietrans{cẹ̄}}} `jazer' > \emph{zī-};
\pac~\ipa{*\emph{\pietrans{cRd-}}} `coração' > luv.cun.\ \emph{zārt-};
\pac~\ipa{*\emph{\pietrans{cwon-}}} `cachorro' > \emph{zuwan-};
\pac~\ipa{*\emph{\pietrans{ecwo-}}} `cavalo' > \emph{azu{(wa)}-}.

A palatal \pac~\ipa{*/\emph{\pietrans{j}}/} e a velar
\ipa{*/\emph{\pietrans{g}}/} se desenvolvem em \ipa{/\emph{y}/} antes de vogais
anteriores e desaparecem antes de \ipa{/\emph{i}/}.
A mudança de \pac~\ipa{*/\emph{\pietrans{j}}/} ou \ipa{*/\emph{\pietrans{g}}/}
para \ipa{/\emph{y}/} por vezes causa elevação da vogal \ipa{/\emph{e}/} para
\ipa{/\emph{i}/} e a consequente queda de \ipa{/\emph{y}/}:
\pac~\ipa{*\emph{\pietrans{jesr-}}} `mão' >
luv.cun.\ \emph{\hittitetrans{īs{(sa)}r{(i)}}} \slash{} luv.hier.\ \emph{istri-}.

\clearpage
