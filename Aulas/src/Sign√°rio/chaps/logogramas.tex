% TeX root=../main.tex

\section{Lista de Logogramas}

A seguir, listam-se os logogramas utilizados ao longo do curso em ordem do
número HH, a partir de~\citet{LarocheHH} com as adequações do \citeabbrev*{CHLI3}.
Utiliza-se a abreviação TO para topônimos, A para antropônimos e TE para teônimos.

{
\large
\begin{longtable}{p{0.06\linewidth}p{0.05\linewidth}p{0.3\linewidth}p{0.4\linewidth}}
	\toprule
	HH  &                          & Leitura                 & Forma subjacente                \\
	\midrule
	\endhead%

	\bottomrule
	\endfoot%

	1   & \luwiantrans{EGO}        & EGO                     & \emph{amu}                      \\
	14  & \luwiantrans{PRAE}       & PRAE                    & \emph{pari, paran}              \\
	17  & \luwiantrans{REX}        & REX                     & \emph{hantawati-}               \\
	45  & \luwiantrans{FILIUS}     & FILIUS                  & \emph{nimuwiza-}                \\
	    &                          & INFANS                  & \emph{nirawani-}                \\
	    &                          & FRATER                  & \emph{lani-}                    \\
	65  & \luwiantrans{PONERE}     & PONERE                  & \emph{tuwa-}                    \\
	85  & \luwiantrans{HALPA}      & HALPA                   & TO, Halpa (=Alepo)              \\
	    &                          & GENUFLECTERE            & Ver combinações                 \\
	102 & \luwiantrans{CERVUS2}    & CERVUS\textsubscript{2} & \emph{Runtiya-}                 \\
	199 & \luwiantrans{TONITRUS}   & TONITRUS                & \emph{Tarhunt{(a)}-}            \\
	212 & \luwiantrans{FLUMEN}     & FLUMEN                  & {-}                             \\
	228 & \luwiantrans{REGIO}      & REGIO                   & {-}                             \\
	231 & \luwiantrans{CASTRUM}    & CASTRUM                 & \emph{harnisa}                  \\
	246 & \luwiantrans{AEDIFICARE} & AEDIFICARE              & \emph{tama-}                    \\
	268 & \luwiantrans{SCALPRUM}   & SCALPRUM                & \emph{katina-}                  \\
	329 & \luwiantrans{REL}        & REL                     & \emph{kwi-, kwa-}, /kwa/, /kwi/ \\
	360 & \luwiantrans{DEUS}       & DEUS                    & \emph{masani-}                  \\
	363 & \luwiantrans{MAGNUS}     & MAGNUS                  & \emph{ura-}                     \\
\end{longtable}
}

\section{Lista de logogramas combinados}

 {
  \large
  \begin{center}
	  \begin{longtable}{lcll}
		  \toprule
		  HH      & Unicode                      & Leitura        & Forma subjacente         \\
		  \midrule
		  \endhead%

		  \bottomrule
		  \endfoot%

		  199+85  & \luwiantrans{TONITRUS-HALPA} & TONITRUS.HALPA & TO, Halpa (=Aleppo)      \\
		  360+199 & \luwiantrans{DEUS-TONITRUS}  & DEUS.TONITRUS  & TE, \emph{Tarhunt{(a)}-} \\
	  \end{longtable}
  \end{center}
 }

