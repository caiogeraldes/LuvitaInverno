% TeX root=../main.tex

\section{Sistema de escrita}

Os hieróglifos anatólicos são um sistema de escrita autóctone da Anatólia
utilizado, até onde se sabe, apenas para escrever textos em luvita.
O sistema utiliza tanto \emph{logogramas}, i.e.\ caracteres que denotam uma
unidade semântica, quanto \emph{fonogramas}, i.e.\ caracteres que denotam sons da
língua.
Há duas variedades principais dos hieróglifos, os de baixo relevo, produzidos
com incisões no material de suportes, e os de alto relevo, produzidos
desbastando a pedra em volta dos caracteres.\footnote{Neste documento,
	caracteres dos hieróglifos anatólicos serão tipografados utilizando a fonte
		{\tiny\texttt{Noto Sans Anatolian Hieroglyphs}}, que os representa no estilo de
	baixo relevo.}
As inscrições do período imperial utilizam sinais levemente diferentes dos
sinais das inscrições do período neo-hitita e seus escribas tendem a preferir
o uso de logogramas em detrimento dos fonogramas.\footnote{%
	Para detalhes do sistema de escrita, vide
	\citet[pp.\ 6ff.\ e pp.\ 23ff.]{CHLI11} e \citet[pp. 354ff.]{CHLI3}.
}


Parte dos hieróglifos pode ter interpretação tanto de logograma quanto de
fonograma e, em alguns casos, a interpretação fonográfica surgiu por
\emph{rebus}, isto é, o logograma passou a ser utilizado para indicar parte do
som da palavra originalmente denotada por ele, como em~\Next.
Alguns sinais não estabilizaram uma leitura fonográfica quando da escrita das
inscrições que nos chegaram e ainda, por vezes, são lidos como \emph{rebus},
como em~\NNext.

\ex.\a. L.66 DARE 𔑈 = \emph{pi{(ya)}-} `dar' $\rightarrow$ /pi/
\b. L.509 (=L.329) CURRERE 𔘰 \slash{} 𔕰 = \emph{hwi{(ya)}-} `correr' $\rightarrow$
/hwi/

\ex. L.13 PRAE 𔐎 = \emph{pari} \slash{} \emph{paran} `em frente' $\rightarrow$
/pa.ri/\footnote{Como no nome próprio Parita, escrito 𔐎𔐞 PRAE-tá- =
	\emph{Parita-} em QALʿAT EL MUDIQ, § 1.}

\paragraph{Transliteração e transcrição}
Por conveniência, costuma-se transliterar o texto hieroglífico no alfabeto latino
e então produzir a transcrição do que se supõe ser a forma ``corrida'' do texto.

\noindent A convenção de transliteração para o alfabeto latino consiste em:
\noprelistbreak%
\begin{compactenum}
	\item Se o sinal não tem interpretação estabelecida ou a interpretação no
	contexto é incerta, incluir o número do logograma conforme em
	\citet{LarocheHH}, seja com um asterisco ou um \emph{L.} antecedendo o
	número
	\item Se o sinal tem valor logográfico ou \emph{rebus},
	escrever o valor semântico convencional em latim,
	seguindo~\citet{LarocheHH} e letras maiúsculas.\footnote{Por vezes,
		sinais que denotam topônimos não são latinizados e são grafados em itálico.}
	\item Se um ou mais logogramas estão em função de \emph{determinativo}
	(\emph{vide sub}), eles são colocados entre parênteses.
	\item Se o sinal tem valor fonográfico, utilizar letras minúsculas.
	\item Sinais que pertencem à mesma palavra são separados por hifens.
\end{compactenum}
A transcrição seguem as seguintes convenções:
\begin{compactenum}
	\item sinais sem interpretação estabelecida ou logogramas cuja forma linguística
	subjacente é desconhecida, permanecem transliterados;
	\item sinais logográficos com interpretação conhecida são convertidos pela
	palavra que representam;
	\item sinais interpretados como \emph{rebus} são convertidos pro valor
	fonológico;
	\item os hifens são excluídos e os sinais com valor fonológico são unidos.
\end{compactenum}
Como a transcrição depende da interpretação das formas linguísticas subjacentes,
a conversão não é de um para um e depende do nossas suposições sobre a língua.
Com frequência, diferentes autores produzem diferentes transcrições para uma mesma
sequência de sinais e, quando em dúvida entre duas formas possíveis, incluem
parênteses nos pontos incertos.

\subsection{Fonogramas}

Os fonogramas dos hieróglifos anatólicos representam unidades de sílabas, sendo
também chamados de silabogramas.
Em sua maioria, eles representam sequências de V (Vogal) e CV (Consoante Vogal),
com alguns poucos representando a sequência CVCV, mas apenas quando a segunda
sequência de \emph{consoante-vogal} representa a sílaba \emph{ra\slash{}ri}.\@
O silabário ``regular'' para o período das citadas-estado neo-hititas está
representado em~\autoref{tab:silabariobasico} e~\autoref{tab:silabariobasicob} e
os sinais para séries CVCV estão em~\autoref{tab:CVCV}.

\paragraph{Fonogramas múltiplos}
Sons que podem ser representadas por mais de um sinal recebem na transliteração
sinais adicionais. Utilizando por exemplo o som /a/, a forma mais comum será
transliterada <a>, a segunda mais comum pelo acento agudo <á> (=a$_2$), a terceira
pelo acento grave <à> (=a$_3$) e as demais por números subscritos, como <a$_5$>.
Formas que podem ter diversas vogais são grafadas com as opções de vogal
separadas por uma barra, < \slash{} >.

\clearpage
{
	\begin{figure}[ht]
		\makebox[\textwidth][l]{
			\includegraphics[width=1.2\textwidth,trim={14mm 30mm 28mm 55mm},clip]{../../Mídia/silaba.pdf}
		}
		\caption{Silabário regular~\cite[419]{CHLI3} -- Parte 1}\label{tab:silabariobasico}
	\end{figure}
}
\clearpage

\clearpage
{
	\begin{figure}[ht]
		\makebox[\textwidth][r]{
			\includegraphics[width=1.2\textwidth,trim={22mm 200mm 27mm 30mm},clip]{../../Mídia/silabario-b.pdf}
		}
		\caption{Silabário regular~\cite[421]{CHLI3} -- Parte 2}\label{tab:silabariobasicob}
	\end{figure}
	\begin{figure}[ht]
		\makebox[\textwidth][c]{
			\includegraphics[width=1.1\textwidth,trim={22mm 149mm 27mm 39mm},clip]{../../Mídia/silabario-c.pdf}
		}
		\caption{Fonogramas CVCV~\cite[422]{CHLI3}}\label{tab:CVCV}
	\end{figure}
}
\clearpage


\paragraph{Consoantes isoladas}
Com esse sistema que sempre representa sequências {(C)}V, é impossível
representar encontros consonantais e consoantes finais.
Via de regra, o costume dos escribas era de grafar uma consoante qualquer X com
o fonograma utilizado para grafar a sílaba /Xa/.
Em português, isso tornaria as palavras \emph{barco} e \emph{barraco} idênticas
na grafia, <ba-ra-co>, exigindo que o falante recuperasse pelo contexto e
conhecimento da língua qual a forma fonológica ali
representada.\footnote{Consoantes geminadas não são representadas nos
	hieróglifos anatólicos.}
Assim, para escrever \emph{hamsukalas} ``bisneto'', um escriba de MARAŞ 1 escreveu:

\exg. \ldots{} \Large 𔓷 \Large 𔒅 \Large 𔖢 \Large 𔗧 \Large 𔓊 \Large 𔗦\\
\ldots{} ha ma su ka la sá\\
\ldots{} \emph{ha\textbf{m}sukala\textbf{s}} (MARAŞ 1, §1d)

Aqui, os grafemas <ma> e <sá> devem ser interpretados como as suas respectivas
consoantes puras /m/ e /s/.

\paragraph{/n/ pré-consonantal}
Uma particularidade da escrita luvita é não grafar o /n/
pré-consonantal onde ele seria esperado pela reconstrução linguística ou
comparação com o luvita cuneiforme.\footnote{%
	Alguns interpretam nisso um sinal de que, ao menos no dialeto das inscrições
	em hieróglifos, os falantes não mais produziam a consoante /n/, mas sim
	a nasalização da vogal anterior, o que não estaria documentado nos textos
	luvitas em cuneiforme por conta ou de práticas ortográficas de escribas
	acostumados com a ortografia cuneiforme do hitita, ou de uma diferença
	dialetal entre o dialeto da era do bronze e da era do ferro. Se for este o
	caso, o exemplo~\ref{ex:tatinzi} representaria /ta.tĩ.tsi/.}
Em português, isso tornaria as palavras \emph{manga} e \emph{maga} idênticas na
grafia, <ma-ga>.
O mesmo escriba de \Last{} para escrever a palavra para `pais' escreve:

\exg.\label{ex:tatinzi}\Large 𔐞 \Large 𔑣 \Large 𔖩\\
tá ti zi\\
\emph{tati\textbf{n}zi}  (MARAŞ 1, §12)

\subsection{Logogramas}

Os logogramas dos hieróglifos anatólicos representam unidades de sentido
completo como palavras ou conceitos que, às vezes, podem ser interpretados pelo
desenho que representam, como em~\ref{ex:ovis}.
As palavras em luvita subjacentes aos logogramas só nos são conhecidas por
ocasiões em que o escriba, além de utilizar o logograma, escreve também a
palavra com os silabogramas da forma, como é o caso em~\ref{ex:ovis-hawa}.

\exg.\label{ex:ovis}\Large 𔒇\\
OVIS\\
-\\
`ovelha' (EMİRGAZİ 1, §§19)

\exg.\label{ex:ovis-hawa}\Large 𔒇 \Large 𔓷 \Large 𔗬\\
OVIS ha wa/i\\
\emph{hawa}\\
`ovelha' (KULULU l.s. 2, §§1.2–11, etc.)


\paragraph{Palavra subjacente desconhecida}
No entanto, não é sempre que temos essa sorte e todas as atestações de um
logograma que nos chegaram o fizeram sem os complementos fonológicos, como
em~\ref{ex:capra}. Nesse caso, sabendo que o sinal L.104 𔑶 CAPRA é utilizado
também para grafar a sílaba /sa/, tanto na forma mais pictórica como na forma
simplificada 𔑷 e que em hitita a palavra para caprinos é \emph{šaš{(š)}a}, podemos
supor que a palavra subjacente ao logograma L.104 é \emph{sasa-}.
Comparações com o luvita cuneiforme e com línguas histórica e
geograficamente próximas do luvita hieroglífico nos permitem elucidar as formas
subjacentes que não nos chegaram grafadas, mas há casos em que é impossível
alcançar qualquer suposição razoável ou satisfatória, como em~\ref{ex:adorare}.

\exg.\label{ex:capra}\Large 𔑶\slash{}𔑷\\
CAPRA\\
\emph{sasa}{ }? (hit. \emph{šaš{(š)}a})\\
`cabra'


\exg.\label{ex:adorare}\Large 𔐅  \\
ADORARE\\
???\\
`rezar?' (HİSARCIK 2, § 1)

\paragraph{Logograma + silabogramas}
Como mencionado acima e ilustrado por~\ref{ex:ovis-hawa}, por vezes um logograma
é seguido da palavra subjacente escrita por completo. Essa prática é comum e
frequente.
Além disso, alguns logogramas são seguidos de silabogramas representando apenas
partes da palavra subjacente.
Por vezes, como em~\ref{ex:ponere-ha}, apenas a desinência flexional da
palavra é escrita (i.e., as marcas de caso, gênero e número para substantivos e
as de número, pessoa, tempo e modo).
Em outros casos, partes além da desinência são escritas com os silabogramas
enquanto outras são deixadas sem representação, como em~\ref{ex:ponere-wa-ta},
em que a primeira sílaba de /tu.wa.ta/, /tu/, é representada pelo logograma,
enquanto as demais sílabas são representadas com silabogramas.

\exg.\label{ex:ponere-ha}\Large 𔑇 \Large 𔓷\\
PONERE ha\\
\emph{tuwaha}\\
`(eu) coloquei' (HAMA 4, §§7)

\exg.\label{ex:ponere-wa-ta}\Large [𔑇] \Large 𔗬 \Large 𔑯\\
PONERE wa/i ta\\
\emph{tuwata}\\
`(ele) colocou' (BABYLON 3)


\noindent Nada exige que as sílabas que seguem um logograma sejam
\emph{contíguas} na palavra subjacente, por vezes apenas a primeira e última
sílabas são representadas.
A palavra para `filho', no nominativo singular, é \emph{nimuwizas}, como
atestado pela escrita plena {(FILIUS)}ni-mu-wa/i-za-sa, bastante frequente no
corpus.\footnote{KÖRKÜN, §1; KARKAMIŠ A2+3, §1; TELL AHMAR 1, §13; EĞREK, §1;
QALʿAT EL MUDIQ, §1; HAMA 4, §1 (-[m]u-); HAMA 8, §1; HINES, §1; ŞIRZI, §1;
KARKAMIŠ A11a, §1; TELL AHMAR 1, §§1, 19(-i).}
No entanto, em algumas inscrições, ela aparece grafada:

\exg.\Large 𔐰 \Large 𔗐 \Large 𔖪 \Large 𔗔\\
FILIUS ni za sa\\
\emph{nimuwizas} \slash{} \emph{nizas}{ }?\\
`filho' (HAMA 1–3, 6–7, §1)

É impossível decidir se a palavra subjacente nesse caso e em situações
semelhantes é uma forma realmente abreviada na fala (uma forma coloquial?)
ou se se trata apenas de uma abreviação gráfica.
Esses casos são, no entanto, raros.

\paragraph{Logogramas com múltiplas leituras}
Alguns logogramas servem para representar múltiplas palavras de um mesmo campo
semântico.
O logograma L.45 𔐰 era utilizado para denotar palavras no campo semântico de
`filho, criança, irmão', sendo transliterada pelas palavras latinas FILIUS,
INFANS e FRATER respectivamente. Nestes casos, é comum que a palavra siga
escrita também em silabogramas, ao menos parcialmente:

\ex.\ag.\Large 𔐰 \Large 𔗐 \Large 𔑿 \Large 𔗬 \Large 𔖪 \Large 𔗔\\
FILIUS ni mu wa\slash{}i za sa\\
\emph{nimuwizas} \\
`filho' (KÖRKÜN, §1)
\bg.\Large 𔐰 \Large 𔗐 \Large 𔗬𔖱 \Large 𔗐\\
INFANS ni wa\slash{}i+ra\slash{}i ni ($=$ INFANS.\emph{NI}-wa/i+ra/i-ni-?)\\
\emph{niwarani}{ }? \\
`criança (incapaz?)' (MARAŞ 4, §14)
\bg.\Large 𔐰 \Large 𔓊 \Large 𔓯 \Large 𔗐\\
FRATER la i sa ($=$ FRATER.\emph{LA}-i-sa?)\\
\emph{lanis}{ }?  ($\simeq$ luv.cun. \emph{nani{(ya)}-}?, cf.\ hit. \emph{negna-})\\
`irmão' (ALEPPO 2, §3)


\noindent Note-se que no caso de \emph{niwarani} `criança' e \emph{lani} `irmão',
não podemos estabelecer certeza da forma fonológica subjacente, posto que ou não
temos esses termos registrados em luvita cuneiforme, o luvita cuneiforme os
registra com variações e a comparação com o hitita
é inconclusiva.\footnote{A interpretação das formas subjacentes ao logograma
	L.45 𔐰 como INFANS e FRATER discutida em~\citet[143--6]{Hawkins1980}
	e~\citet[387]{Yakubovich2010b}.}
O mesmo ocorre em diversos casos em que um logograma possui múltiplas leituras
possíveis.''

\section{Fonologia}

Utilizando apenas o silabário regular do luvita hieroglífico seríamos capazes de
reconstruir o seguinte inventário de fonemas:

\begin{compactitem}
	\item Vogais: \emph{a}, \emph{i}, \emph{u}
	\item Oclusivas: \emph{p}, \emph{t}, \emph{k}
	\item Nasais: \emph{m}, \emph{n}
	\item Fricativas: \emph{s}, \emph{z}, \emph{h}
	\item Outras: \emph{r}, \emph{l}, \emph{w}, \emph{y}
\end{compactitem}
No entanto, esse inventário de fonemas não parece ser o inventário realmente
utilizado pela língua.

Em primeiro lugar, temos como evidência o \emph{rotacismo} de algumas dentais em
ambiente intervocálico.
O rotacismo não ocorre de maneira consistente em nenhuma região geográfica ou
período da língua luvita e formas com os sinais para /t/ e /r/ com frequência
aparecem no mesmo texto.
As formas que sofrem rotacismo são, de acordo
com~\citet[249--50]{MorpurgoDavies1982}:

\begin{compactenum}
	\item desinências de ablativo em \emph{-ati}: <\emph{-a-ti}> e
	<\emph{-Ca-ra\slash{}i-{(i)}}>.
	\item desinências de terceira pessoa:
	\begin{compactenum}
		\item presente \emph{-ti}: <\emph{-ti}> ou <\emph{-ra\slash{}i-{(i)}}>;
		\item pretérito \emph{-ta}: <\emph{-ta}> ou <\emph{-ra\slash{}i}>;
		\item imperativo \emph{-tu}: <\emph{-tu}> ou <\emph{-ru}>.
	\end{compactenum}
	\item partículas enclíticas:
	\begin{compactenum}
		\item reflexivo / pronominal \emph{=ti}: <\emph{-ti}> ou <\emph{-ra\slash{}i-{(i)}}>;
		\item pronominal \emph{=tu}: <\emph{-tu}> ou <\emph{-ru}>;
		\item pronominal \emph{=ata}: <\emph{-a-ta}> ou <\emph{-a+ra\slash{}i}>.
	\end{compactenum}
	\item itens lexicais, dois deles com etimologia bem estabelecida:
	\begin{compactenum}
		\item <\emph{á-ru-na}> `comer' de \textsc{pie} *\emph{ed-};
		\item <\emph{pa+ra\slash{}i-za}> (dat.pl., SULTANHAN, §9) de
		\textsc{pie} *\emph{ped-}, mas também grafado <\emph{pa-da}> (dat.sg.,
		SULTANHAN, §6).
	\end{compactenum}
\end{compactenum}
Por comparação com a evidência do luvita cuneiforme\footnote{Em luvita
	cuneiforme, embora irregular, a consoante /d/ é representada pela grafia das
	dentais sem geminação, como \emph{a-a-ta} /ada/ `ele fez', em contraste com
	\emph{a-at-ta} /ata/ (conj.+partic.).}
e lício e por razões tipológicas, assume-se atualmente que o rotacismo apenas
incindia sobre uma consoante próxima de uma oclusiva sonora dental
/d/.\footnote{Esse /d/ pode ter duas origens, como
	argumenta~\textcite{MorpurgoDavies1982}:
	\begin{inparaenum}
		\item \textsc{pie} *\emph{d};
		\item \textsc{pie} *\emph{t} em pelo menos dois contextos:
		\begin{inparaenum}
			\item \textsc{pie} V̄́tV > luv.com.\ V̄́dV;\@
			\item \textsc{pie} V́CVtV > luv.com.\ V́CVdV;\@
		\end{inparaenum}
	\end{inparaenum}
	i.e., após vogais longas ou ditongos acentuados e entre vogais não acentuadas.
}
Além disso, os sinais previamente considerados intercambiáveis da série
/ta/, antigos <ta\textsubscript{1-5}>, deixaram de sê-lo desde o artigo
de~\citet{Rieken2008}, que demonstrou que <ta\textsubscript{3}> não é
intercambiável com <ta\textsubscript{1--2}> e desde o artigo
de~\citet{RiekenYakubovich2010}, que demonstrou que os antigos
<ta\textsubscript{4}> e <ta\textsubscript{5}> correspondem a <la\slash{}i> e
<lá\slash{}í>, respectivamente.

% existem cinco grafemas utilizados para representar a série
% não são completamente intercambiáveis.
% O segundo grupo, <tà\slash{}ta\textsubscript{3}> <ta\textsubscript{4} e <ta\textsubscript{5}>, representaria
% um fonema que não estava sujeito a \emph{rotacismo}, enquanto o primeiro sim, o
% que~\citet{MorpurgoDavies1982} considera evidência de que esse grupo representa
% apenas oclusivas dentais surdas /t/, enquanto o grupo
% \emph{ta}\textsubscript{1-2} representa as oclusivas dentais sonoras

Em segundo lugar, por comparação com o luvita cuneiforme, cujo sistema de
escrita diferencia oclusivas surdas e sonoras (excluindo as dentais) e é capaz
de representar vogais longas (pela \emph{scriptio plena}), temos evidências de
que o sistema opunha oclusivas \emph{surdas} e \emph{sonoras}, bem como
vogais \emph{breves} e \emph{longas}.

Por fim, o

\section{Flexão nominal}

\lipsum[1]

\clearpage

\section{Leitura: BABYLON 3}

Trata-se de um vaso em estado fragmentário (\autoref{fig:babylon3}) escavado por
Koldewey na década de 20 onde se acredita ser a cidade de Babilônia, sítio
arqueológico de Arpada, noroeste de Alepo, contendo uma inscrição no
beiral em cursivas de baixo relevo, sentido sinistroverso, em duas linhas a
serem lidas em conjunto (para cada coluna, lê-se o caractere na primeira linha,
em seguida o da segunda linha e assim sucessivamente).
A inscrição, embora escavada na Babilônia, provavelmente teria sido
produzida em Alepo e lá dedicada ao deus do trovão Tarhunta da cidade, o que é
indicado pelo epíteto 𔓢𔑞𔕸𔗐 \textsc{tonitrus.halpa}-\textit{pa-ni}
= \textit{halpa{(wa)}ni} `halabeu', desde o período imperial,
a combinação dos logogramas L.199+L.84\slash{}85 𔓢𔑝\slash{}𔑞
\textsc{tonitrus+crus$_2$\slash{}genuflectere}
via de regra denota a cidade de Halab.\footnote{Com L.84
\textsc{crus}$_2$ 𔑝: ALEPPO 5; NİŞANTEPE 2, no. 57; İMAMKULU.\@
Com L.85 \textsc{genuflectere} 𔑞:  ALEPPO 6; TELL AHMAR 5;
KÖRKÜN;\@ BABYLON 1; BABYLON 3; \mbox{HAMA 1}.\@
Também nomes próprios de figuras associadas a Halab são grafados com essa
combinação, como Halparuntiya em
MARAŞ 1 \textsc{tonitrus.halpa}-\textit{pa-ru-ti{(-i)}-ia-}.
}
A data de produção é incerta, mas deve cair entre o século IX e VIII
\textsc{aec}.

\vfill

\begin{figure}[!htb]
	\includegraphics[width=\textwidth,trim={4mm 2mm 12mm 11mm},clip]{../../Mídia/aleppo23.jpg}
	\caption{
		Babylon 3. Diâmetro: 0.66m.; Profundidade
		(interna): 0.67m. Imagens produzidas e traçado feito por
		\citet[\emph{plate} 212]{CHLI13}. Atualmente no
		\foreignlanguage{german}{Vorderasiatisches Museum}, Berlin,
		no. VA Bab. 1507.
	}\label{fig:babylon3}
\end{figure}

\clearpage

\begin{center}
	\includegraphics[width=\textwidth,trim={100mm 60mm 55mm 10mm},clip]{../../Mídia/babylon3.png}
\end{center}
\exg.{\Large 𔖪𔓱𔗬𔗷} {\Large 𔗎𔔯𔗏𔗧𔑣𔐤} {\Large 𔑵𔑣𔓱𔗔} {\Large 𔓢𔑞𔕸𔗐} {\Large 𔖖𔓢𔕙𔑣}
{\Large 𔐎𔐤} {\Large [𔑇]𔗬𔑯}\\
za-ia-wa\slash{}i-a\hspace{10pt} ``SCALPRUM''-ka-ti-na\hspace{10pt}
CERVUS\textsubscript{2}-ti-ia-sa\hspace{10pt} TONITRUS.HALPA-pa-ni\hspace{10pt}
DEUS.TONITRUS-hu-ti\hspace{10pt} PRAE-na\hspace{10pt} [PONERE]-wa/i-ta\\
zaya=wa katin{(a)} Runtiyas halpawani Tarhu{(n)}ti paran tuwa-ta

\exg.zaya=wa katin{(a)} Runtiyas halpawani Tarhu{(n)}ti paran tuwata\\
\Det{}\Acu{} vasilha.\Neut{}\Acu{} R.\Com{}\Nom{} halabeu.\Dat{} T.\Dat{} \Prep{} colocar-3\Sg\\
Esta vaso Runtiyas dedicou ao Tarhunta halabeu.


