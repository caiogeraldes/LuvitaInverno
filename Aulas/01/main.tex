\documentclass[a4paper,12pt,article]{memoir}

\usepackage{csquotes}
\usepackage{fontspec}
\usepackage[main=brazil]{babel}
% \defaultfontfeatures{Renderer=Harfbuzz}

\babelfont[brazil]{rm}[
	Numbers=OldStyle,
	SmallCapsFont=Gentium Plus,
	SmallCapsFeatures={Letters=SmallCaps}]{Crimson Pro}
\babelfont[brazil]{sf}{Noto Sans}
\babelfont[brazil]{tt}{Mononoki Nerd Font}

\babelprovide[import, onchar=ids fonts letters]{ancientgreek}
\babelfont[ancientgreek]{rm}{Brill}
\babeltags{grc = ancientgreek}

\input{luwian.tex}

\babelprovide{hittite}
\babelfont[hittite]{rm}{UllikummiA}

\babelprovide[import,onchar=ids fonts]{sanskrit}
\babelfont[sanskrit]{rm}[Scale=1]{Noto Serif Devanagari}
\babelfont[sanskrit]{sf}[Scale=1]{Noto Sans Devanagari}

\usepackage{hyperref}%

\title{Luvita Hieroglífico -- Aula 01}
\author{Caio Geraldes }

\usepackage[backend=biber,
	style=abnt,
	repeatfields,
	ittitles,
	% indent,
	giveninits,
	justify,
	noslsn,
	natbib,
	extrayear,
]{biblatex}
\addbibresource{../../Bibliografia/biblio.bib}


\renewcommand{\maketitle}{
    \noindent{\huge\thetitle}
    \vspace{0.75\baselineskip}

 \noindent{\Large{}Gramática: Sistema de escrita, flexão nominal}

    \vspace{0.25\baselineskip}

    \noindent{\Large Leitura: BABYLON 3}

    \vspace{0.75\baselineskip}

    \noindent{\large\theauthor} -- \href{mailto:caio.geraldes@usp.br}{\texttt{<caio.geraldes@usp.br>}}

    \vspace*{2\baselineskip}
    \thispagestyle{title}
}

\usepackage{lipsum}

\begin{document}
\maketitle


\chapter{Introdução: quem, quando e onde?}


\textbf{Luvita} denota um povo e uma língua e seus dialetos cuja existência,
até o começo do século passado, estava perdida na história.\footnote{Esta seção
	está baseada sobretudo em \textcite{CHLI_1_1,Melchert2003,Hoffner2008}.}
Quando no final do século \textsc{xix} foram encontrados blocos de pedra no
norte da Síria com inscrições em hieroglifos em alto relevo, associaram esta
nova língua e o povo que a escreveu com os \textbf{hititas}, um povo que até
então era lembrado por passagens da bíblia hebraica e alguns documentos
recentemente descobertos em assírio.
Em 1906, as escavações realizadas em Boğazköy\slash{}Boğazkale sob deiração de
Hugo Winckler e Theodore Makridi revelaram a cidade de Hattusa,
capital do que teria sido depois chamado de Império Hitita, e nela um grande
arquivo de documentos em cuneiforme em uma língua até então
desconhecida.\footnote{A decifração do cuneiforme nesta altura já estava
	bastante adiantada, tendo sido iniciada nos primeiros anos do século
	\textsc{xix} e relativamente bem estabelecida dentro da primeira metade do
	século para o persa antigo, acadiano e elamita.}
Apenas em 1915-17, Bedřich Hrozný conseguiria ao mesmo tempo demonstrar que a
língua nesses arquivos e em duas cartas previamente escavadas em
\href{https://pleiades.stoa.org/places/149576487}{Tell el-Amarna}
(Egito moderno) era uma língua indo-europeia e produzir um esboço gramatical
dela, identificando-a como a língua dos hititas.
A língua dos hieróglifos sírios passou a ser conhecida por ``hitita
hieroglífico'' e, de sua descoberta até a década de 20, os textos permaneceram
praticamente ilegíveis.\footnote{Alguns sinais tinham sido corretamente
	interpretados por Sayce entre 1882 e 1884, a saber os logogramas L.17 𔐑
	\textsc{rex} e L.228 𔔆 \textsc{regio}, respectivamente correspondentes aos
	cuneiformes \foreignlanguage{hittite}{𒈗} \textsc{lugal} `rei' e
	\foreignlanguage{hittite}{𒆳} \textsc{kur} `país\slash{}território'.}


\printbibliography%

\end{document}
