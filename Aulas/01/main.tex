\documentclass[a4paper,12pt]{memoir}

\settrimmedsize{\stockheight}{\stockwidth}{*}
\settypeblocksize{230mm}{130mm}{*}
\setlrmargins{*}{*}{1.1}
\setulmargins{30mm}{*}{*}
\checkandfixthelayout%

\setlength{\footmarkwidth}{0.5em}
\setlength{\footmarksep}{0em}
\setlength{\footparindent}{0em}
\footmarkstyle{\textsuperscript{#1}\hspace{0.5em}}

\makeoddfoot{plain}{}{}{\thepage}
\makeevenfoot{plain}{\thepage}{}{}
\makepagestyle{ruled}
\makeevenfoot{ruled}{\thepage}{}{} % page numbers at the outside
\makeoddfoot{ruled}{}{}{\thepage}
\makeheadrule{ruled}{\textwidth}{0.75pt}
\makeevenhead{ruled}{\scshape\leftmark}{}{}
\makeoddhead{ruled}{}{}{\scshape\rightmark}
\makepsmarks{ruled}{%
        \nouppercaseheads
        \createmark{chapter}{left}{shownumber}{\scshape}{.\space}
        \createmark{part}{right}{shownumber}{}{.\space}
        \createmark{section}{right}{shownumber}{}{.\space}
        \createmark{subsection}{right}{shownumber}{}{.\space}
        \createplainmark{toc}{both}{\contentsname}
        \createplainmark{lof}{both}{\listfigurename}
        \createplainmark{lot}{both}{\listtablename}
        \createplainmark{bib}{both}{\bibname}
        \createplainmark{index}{both}{\indexname}
        \createplainmark{glossary}{both}{\glossaryname}
}

% Decorando divisões
\chapterstyle{hangnum}
\setsecnumdepth{subsection}
\setcounter{tocdepth}{3}
\newcommand\chap[1]{%
  \chapter*{#1}%
  \addcontentsline{toc}{chapter}{#1}}


\usepackage{paralist}
\usepackage{graphicx}
\usepackage{hyphenat}
\usepackage{linguex}

% Paleta de cores
\usepackage{xcolor}
\definecolor{green}{RGB}{16,87,87} % rgb(16,87,87)
\definecolor{red}{RGB}{193, 11, 105} % rgb(193, 11, 105)
\definecolor{yellow}{RGB}{218,222,104} % rgb(218,222,104)
\definecolor{pink}{RGB}{243,179,145} % rgb(243,179,145)
\definecolor{blue}{RGB}{161,184,206} % rgb(161,184,206)

\usepackage{hyperref}%
\hypersetup{%
        colorlinks=true, % false: boxed links; true: colored links
        linkcolor=green,  % color of internal links
        citecolor=green,  % color of links to bibliography
        filecolor=pink,  % color of file links
        urlcolor=green,
        bookmarksdepth=4
}

\newcommand{\Prep}{\footnotesize\textsc{Prep.}}
\newcommand{\Det}{\footnotesize\textsc{Det.}}
\newcommand{\Clt}{\footnotesize\textsc{Clt.}}
\newcommand{\Nom}{\footnotesize\textsc{Nom.}}
\newcommand{\Acu}{\footnotesize\textsc{Acu.}}
\newcommand{\Dat}{\footnotesize\textsc{Dat.}}
\newcommand{\Gen}{\footnotesize\textsc{Gen.}}
\newcommand{\Abl}{\footnotesize\textsc{Abl.}}
\newcommand{\Sg}{\footnotesize\textsc{Sg.}}
\newcommand{\Pl}{\footnotesize\textsc{Pl.}}
\newcommand{\Com}{\footnotesize\textsc{Com.}}
\newcommand{\Neut}{\footnotesize\textsc{Neut.}}

\usepackage{csquotes}
\usepackage{fontspec}
\usepackage[main=brazil]{babel}
% \defaultfontfeatures{Renderer=Harfbuzz}

\babelfont[brazil]{rm}[
	SmallCapsFont=Gentium Plus,
	SmallCapsFeatures={Letters=SmallCaps}]{Crimson Pro}
\babelfont[brazil]{sf}{Noto Sans}
\babelfont[brazil]{tt}{Mononoki Nerd Font}

\babelfont[german]{rm}[
	SmallCapsFont=Gentium Plus,
	SmallCapsFeatures={Letters=SmallCaps}]{Crimson Pro}
\babelfont[brazil]{sf}{Noto Sans}
\babelfont[brazil]{tt}{Mononoki Nerd Font}

\babelfont[english]{rm}[
	SmallCapsFont=Gentium Plus,
	SmallCapsFeatures={Letters=SmallCaps}]{Crimson Pro}
\babelfont[brazil]{sf}{Noto Sans}
\babelfont[brazil]{tt}{Mononoki Nerd Font}

\hyphenation{
	hi-e-ro-glí-fi-co
}

\babelprovide[import, onchar=ids fonts letters]{ancientgreek}
\babelfont[ancientgreek]{rm}{Brill}
\babeltags{grc = ancientgreek}

\input{luwian.tex}

\babelprovide{hittite}
\babelfont[hittite]{rm}{UllikummiA}

\babelprovide[import,onchar=ids fonts]{sanskrit}
\babelfont[sanskrit]{rm}[Scale=1]{Noto Serif Devanagari}
\babelfont[sanskrit]{sf}[Scale=1]{Noto Sans Devanagari}


\title{Luvita Hieroglífico}
\author{Caio Geraldes}

\usepackage[backend=biber,
	style=abnt,
	repeatfields,
	ittitles,
	indent,
	giveninits,
	justify,
	noslsn,
	natbib,
	extrayear,
]{biblatex}
\addbibresource{../../Bibliografia/biblio.bib}

\renewcommand{\maketitle}{
 \thispagestyle{empty}
 \vfill{}
 \noindent{\huge\thetitle}
 \vspace{0.75\baselineskip}

 \noindent{\Large{}Gramática e leitura}

 \vspace{0.75\baselineskip}

 \noindent{\large\theauthor} -- \href{mailto:caio.geraldes@usp.br}{\texttt{<caio.geraldes@usp.br>}}

 \vfill{}
}

\usepackage{lipsum}

\pagestyle{companion}

\newcommand{\colofao}[0]{
    \pagestyle{empty}
    \cleartoevenpage%
    \hfill
    \vfill
        \begin{adjustwidth}{5em}{5em}
            \begin{center}
            \noindent Esse documento foi diagramado usando o sistema
						\href{https://lualatex.org}{Lua\TeX} mantido por Manuel
						Pégourié-Gonnard. Todos os \emph{softwares} utilizados na
						diagramação deste document são gratuitos e \emph{open source}.\par
            {\today.}
            \end{center}
        \end{adjustwidth}
    \bigskip
    \noindent
}


\begin{document}

\setlength{\Exlabelsep}{0.5em}
\setlength{\SubExleftmargin}{1.5em}
\frontmatter
\maketitle

\clearpage
\tableofcontents*

\mainmatter%


\chap{Introdução}

\section{Quem, quando e onde?}

\emph{Luvita} denota um povo e uma língua e seus dialetos cuja existência,
até o começo do século passado, estava perdida na história.\footnote{Esta seção
	está baseada sobretudo em \textcite{CHLI_1_1,Melchert2003,Hoffner2008}.}
Quando no final do século \textsc{xix} foram encontrados blocos de pedra no
norte da Síria com inscrições em hieroglifos em alto relevo, associaram esta
nova língua e o povo que a escreveu com os \emph{hititas}, um povo que até
então era lembrado por passagens da bíblia hebraica e alguns documentos
recentemente descobertos em assírio.
Em 1906, as escavações realizadas em Boğazköy\slash{}Boğazkale sob deiração de
Hugo Winckler e Theodore Makridi revelaram a cidade de Hattusa,
capital do que teria sido depois chamado de Império Hitita, e nela um grande
arquivo de documentos em cuneiforme em uma língua até então
desconhecida.\footnote{A decifração do cuneiforme nesta altura já estava
	bastante adiantada, tendo sido iniciada nos primeiros anos do século
	\textsc{xix} e relativamente bem estabelecida dentro da primeira metade do
	século para o persa antigo, acadiano e elamita.}
Apenas em 1915-17, Bedřich Hrozný conseguiria ao mesmo tempo demonstrar que a
língua nesses arquivos e em duas cartas previamente escavadas em
\href{https://pleiades.stoa.org/places/149576487}{Tell el-Amarna}
(Egito moderno) era uma língua indo-europeia e produzir um esboço gramatical
dela, identificando-a como a língua dos hititas.
Entre os textos em cuneiforme escavados em Boğazköy  entre 1906 e 22 revelaram
dentro deles uma outra língua que viria a ser conhecida pelo nome de
\emph{luvita}.\footnote{Alguns termos soltos dessa língua aparecem marcados com
	um sinal cuneiforme, \foreignlanguage{hittite}{𒃵}, chamado pelo nome alemão
	\emph{Glossenkeil} em meio a textos hititas.}

A língua dos hieróglifos das inscrições sírias, no entanto, permaneceu
praticamente ilegível desde sua descoberta até a década de 30.\footnote{Alguns
	sinais tinham sido corretamente interpretados por Sayce entre 1882 e 1884, a
	saber os logogramas L.17 𔐑 \textsc{rex} e L.228 𔔆 \textsc{regio},
	respectivamente correspondentes aos cuneiformes \foreignlanguage{hittite}{𒈗}
	\textsc{lugal} `rei' e \foreignlanguage{hittite}{𒆳} \textsc{kur}
	`país\slash{}território'.}
No começo da década de 30, contribuições separadas de Meriggi, Gelb, Forrer,
Bossert e Hrozný ofereceram interpretação de diversos logogramas e
interpretações ou, ao menos, aproximações para alguns silabogramas,
permitindo as primeiras tentativas de interpretação.
No começo da década de 40, Güterbock compila selos contendo escrita cuneiforme e
hieroglífica e avança na interpretação dos grafemas.
Entre 1946 e 1960, Bossert passa a publicar a inscrição bilíngue em hieróglifos
e em fenício descoberta por ele e Halet Çambel em Karatepe.

\noindent Datas:
\begin{compactenum}
	\item Imperial: séc. \textsc{xiii aec}, entre as dinastinas de Tudhaliya IV e
	Suppiluliuma II
	\item Neo-hitita: \emph{circa} 1100-700 \textsc{aec}
\end{compactenum}

\begin{figure}[ht!]
	\begin{center}
		\includegraphics[width=1\textwidth]{../../Mídia/Map01.png}
	\end{center}
	\caption{Mapa contendo a localização das inscrições monumentais em luvita
		hieroglífico. Os pontos laranjas representam inscrições do período imperial
		enquanto os verdes, inscrições do período neo-hitita.}\label{fig:mapa1}
\end{figure}

\chapter{Sistema de escrita, fonologia e flexão nominal}

\section{Sistema de escrita}

\lipsum[4]

\section{Fonologia}

\lipsum[3]

\section{Flexão nominal}

\lipsum[1]

\section{Leitura: BABYLON 3}

Trata-se de um vaso em estado fragmentário (quatro pedaços) escavado por
Koldewey na década de 20 onde se acredita ser a cidade de Babilônia, sítio
arqueológico de Arpada, noroeste de Aleppo (Síria), contendo uma inscrição no
beiral em cursivas de baixo relevo, sentido sinistroverso, em duas linhas a
serem lidas em conjunto (para cada coluna, lê-se o caractere na primeira linha,
em seguida o da segunda linha e assim sucessivamente).

\begin{figure}[ht!]
	\begin{center}
		\includegraphics[width=\textwidth]{../../Mídia/aleppo23.jpg}
	\end{center}
	\begin{center}
		\includegraphics[width=\textwidth]{../../Mídia/babylon3.png}
	\end{center}
	\caption{Babylon 3. Diâmetro: 0.66m.; Profundidade (interna): 0.67m. Imagens
		produzidas e traçado feito por \textcite[\emph{plate} 212]{CHLI_1_3}.
		Atualmente no \foreignlanguage{german}{Vorderasiatisches Museum}, Berlin, no. VA Bab. 1507.}\label{fig:babylon3}
\end{figure}


\ex.\ag.{\Large 𔖪𔓱𔗬𔗷} {\Large 𔗎𔔯𔗏𔗧𔑣𔐤} {\Large 𔑵𔑣𔓱𔗔} {\Large 𔓢𔑞𔕸𔗐} {\Large 𔖖𔓢𔕙𔑣}
{\Large 𔐎𔐤} {\Large [𔑇]𔗬𔑯}\\
za-ia-wa/i-a\hspace{10pt} ``SCALPRUM''-ka-ti-na\hspace{10pt}
CERVUS\textsubscript{2}-ti-ia-sa\hspace{10pt} TONITRUS.HALPA-pa-ni\hspace{10pt}
DEUS.TONITRUS-hu-ti\hspace{10pt} PRAE-na\hspace{10pt} [PONERE]-wa/i-ta\\
zaya=wa katin(a) Runtiyas halpawani Tarhu(n)ti paran tuwa-ta
\bg. zaya=wa katin(a) Runtiyas halpawani Tarhu(n)ti paran tuwata\\
\Det{}\Acu{} vasilha.\Neut{}\Acu{} R.\Com{}\Nom{} halabeu.\Dat{} T.\Dat{} \Prep{} colocar-3\Sg\\
Esta vaso Runtiyas colocou em frente (=dedicou) ao Tarhunta halabeu.


\backmatter%


\chap{Vocabulário}

{
	\normalsize
	kati- (subst. \Neut) vaso, vasilha\\
	Runtiya- (nome próprio. \Com) Runtiya\\
	halpawan- (adj.) proveniente de Halpa; halabeu\\
	Tarhu(n)t- (nome próprio, teônimo) Tarhunta\\
	paran (\Prep) em frente a\\
	tuwa- (v.)\\
}

\chap{Glossário}

\nocite{*}

\printbibliography%

\colofao{}

\end{document}
