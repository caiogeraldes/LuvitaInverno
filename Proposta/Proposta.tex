\documentclass[article,12pt]{memoir}

\usepackage{hyperref}

\usepackage[main=brazil]{babel}
\usepackage{fontspec}
\setmainfont{Brill}

\usepackage{paralist}

\title{Introdução ao Luvita Hieroglífico:\\\large Panorama gramatical e inscrições da idade do ferro}
\author{Caio Geraldes\\\url{caio.geraldes@usp.br}}
\date{\today}

\begin{document}

\maketitle

\noindent\textbf{Ministrante:} Caio Geraldes

\noindent\textbf{Coordenador:} José Marcos Macedo

\noindent\textbf{Departamento:} DLCV

\chapter{Introdução}

O luvita (en. \emph{Luwian}) é uma língua indo-europeia antiga do ramo
anatólico atestada em dois sistemas de escrita ao logo da idade do
Bronze e do Ferro. Um dos \emph{dialetos} atestados está registrado em
um sistema de escrita hieroglífico autóctone e sem relação aparente com
os sistemas de escrita do oriente próximo (variedades de cuneiforme e
fenício). Este curso pretende oferecer uma breve introdução à língua e
leitura guiada de documentos luvitas da idade do Ferro, oferecendo um
ponto de partida ao estudo de línguas anatólicas e da história do
período dos estados neo-hititas.

\chapter{Objetivos}

Apresentar o panorama da gramática do luvita hieroglífico, lingua
indo-europeia do ramo anatólico atestada em selos, inscrições
monumentais, dedicatórias e cartas na era do Bronze e do Ferro em um
sistema de escrita autóctone. Conduzir leituras de inscrições da idade
do Ferro e dos estados neo-hititas em luvita, oferecendo ponto de
partida e aparato metodológico aos alunos para a investigação das
questões relativas à interpretação linguística e histórica dos textos
supérstites.

\chapter{Justificativa}

Os documentos luvitas em inscrições monumentais da idade do Ferro, do
ponto de vista linguístico, registram uma língua extremamente próxima
formal, geográfica e historicamente do hitita, oferecendo um novo termo
de comparação para melhor compreender a linguística histórica do ramo
anatólico; e, do ponto de vista histórico, registram eventos posteriores
à dissolução do império hitita, incluindo também marcas dos contatos
populacionais entre povos falantes de línguas indo-europeias da Anatólia
com povos falantes de línguas semíticas (acádios e fenícios) e com os
hurritas. Os métodos empregados na interpretação dos textos de línguas
fragmentárias que serão expostos neste curso também são de interesse,
posto que servem também ao trabalho em outros contextos históricos e
geográficos, tanto na pesquisa linguística quanto na pesquisa histórica.


\chapter{Calendário de aulas}

\textbf{Total:} 10 horas (5 aulas de 2 horas)

\begin{compactitem}
	\item Primeira semana:
	\begin{compactitem}
		\item Apresentação da língua, contemplando informações históricas e
		geográficas relevantes e breve apresentação das circunstâncias do
		descobrimento dos documentos e deciframento do sistema de escrita.
		(1h)
		\item Tópicos da gramática do dialeto hieroglífico do luvita (45m)

		\begin{compactitem}
			\item Sistema de escrita e fonologia (30m)
			\item Flexão nominal (15m)
		\end{compactitem}
		\item Leitura da dedicatória Babylon 3 (15m)
	\end{compactitem}
	\item Segunda semana:

	\begin{compactitem}
		\item Tópicos da gramática do dialeto hieroglífico do luvita (1h):
		\begin{compactitem}
			\item Flexão verbal (30m)
			\item Clíticos (30m)
		\end{compactitem}
		\item Leitura da inscrição HAMA 2 (1h)
	\end{compactitem}
	\item Terceira semana:
	\begin{compactitem}
		\item Tópicos da gramática do dialeto hieroglífico do luvita (1h):
		\begin{compactitem}
			\item Sintaxe (1h)
		\end{compactitem}
		\item Leitura da inscrição BOHÇA (1h)
	\end{compactitem}
	\item Quarta semana:
	\begin{compactitem}
		\item Tópicos da gramática do dialeto hieroglífico do luvita (1h):
		\begin{compactitem}
			\item Questões linguísticas: mutação em -i-, linguística comparada (1h)
		\end{compactitem}
		\item Leitura da inscrição KARKAMISH A11b+c (1h)
	\end{compactitem}
	\item Quinta semana: Leitura de trechos da inscrição de KARATEPE (2h)
\end{compactitem}

\end{document}
